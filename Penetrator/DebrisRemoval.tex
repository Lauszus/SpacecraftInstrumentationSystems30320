\autsubsection{Debris Removal}{Lukas Christensen}
Experiments with ice melting probes on Earth have shown that, while they work in principle, they tend to fail after a while as debris such as sand and dust pile up in front of them\cite{article:di1998}. Even in very pure ice this still becomes a problem as the debris from the entire ice column accumulate in the bottom of the bore hole as the drilling process progresses. 

\subsubsection{Debris Types}
As the exact makeup of the Europan ice sheet is still unknown it is hard to find resources detailing the types of debris that can be expected to be found during the melting process. However, at least three sources can be easily identified:\\

\begin{itemize}
	\item Surface depositions from extra-europan sources (e.g. volcanic dust from Io).
	\item Particles that were suspended in the water when the ice was formed.
	\item Meteorites.
\end{itemize}



\subsubsection{Removal Options}

\subsubsection{Future Work}