\autchapter{Introduction}{Jean-Paul Breuer}
% What is our purpose for the mission. Why? How will we achieve our mission objectives? Water, energy, nutrition
% Estimate:
% Mass, Volume, Power budgets
% Why are we going?
% Are we alone?
% Water, energy, nutritions
% Use pasos
    
% General info: https://en.wikipedia.org/wiki/Colonization_of_Europa

\section{Purpose of the Mission}
Throughout the billions of years of Earth geologic history, it still took several hundreds of millions of years before evolution allowed for complex, intelligent organisms to inhabit the Earth. As soon as free thinking and self-reflection was conceivable in animals, naturally some of the first questions were the big philosophical ones: What is the origin of life? What is the nature of reality and the universe? What are the conditions necessary to support and develop life? Are we alone in the universe?

Being grounded in philosophy, some of these questions might never be answered, and despite the many advancements in technology, the few that could have definite answers still manage to elude us. Perhaps it is the infinite size of the universe that creates problems with communication and observation over such large scales, or perhaps the conditions necessary for intelligent life are so rare that the likely-hood of finding intelligent life within a lifetime is close to null. Either way, the search for life continues, and given the nature of infinity, it becomes not such a question of if these organisms exist, but more a matter of when we will encounter them.

Realistically, finding intelligent life is a `moon shot'; however, the possibility of finding bacteria or less advanced organisms could definitely be promising. Even finding one extremophilic Archaebacteria on a foreign celestial body would have drastic consequences for Earth as we know it. How would society process this information, knowing that we are no longer alone in the universe? What might we be able to learn from foreign biology that could improve our current quality of life? By studying these organisms, perhaps it might be understood where the origin of life came from, or perhaps it might help to answer one of many other larger questions.

\subsection{Where, Why, and How}
Naturally, the best place to search for alien life would be at the most probable locations where one would expect life. Assuming a dependency on water, nutrients, and energy, planets within the habitable zone of any star system would be the most ideal candidates where one would expect to find results. At this point in time, it is not reasonable to reach foreign stars; however, keeping the assumption on the biological dependencies, and restricting the distance parameter to our local neighborhood, the most likely location of finding life within our solar system would be on Jupiter's icy moon Europa. Of course, this would be a very complex mission to undertake with several underlying concepts that would be needed to take into account.

There are many icy moons throughout the solar system, but when it comes to the objective of finding life, Europa is unique in that it has a subsurface ocean under the ice layer. It is theorized that the strong gravitational pull of Jupiter causes a tidal wave, which induces tidal friction and enough energy that internally heats the moon, maintaining the water at a liquid state. Since the recipe for life as we know it requires water, nutrients, and energy, it is hypothesized that Europa might have sufficient conditions. The mission would be split into 6 main phases: the Orbiter phase would involve the launch, coasting, and orbital maneuvers to put a spacecraft on an orbit around Europa; following this is an analysis phase where an ideal landing site would be chosen based on the observed data from the Orbiter; the Lander phase would involve the rapid descent from the orbiter onto the surface of Europa near the ideal landing site; the Penetrator phase would involve the descent and anchoring of the instrument suite as well as the placement of communication relays throughout the ice column; whereby the Instrument Suite would take samples and perform experiments; and finally would have a series of data transfers up through the relays to the Lander, to the Orbiter, and finally back to Earth.

This report will elaborate on a mission concept for a mission to Europa with the objective of finding life. The report has been split into several components. Chapter 2 will formulate the problems and limitations with the mission plans. Chapter 3 will provide an in depth overview of the theory and assumptions made for the mission, including further characteristics of the ice and moon, a definition of what exactly is meant by `life', information on the imaging and telescope systems, and a more in depth overview on why Europa was selected as the most viable solar system candidate. Chapter 4 will describe the Strawman mission in which the remaining chapters will be based upon.  Chapter 5 will elaborate on the Orbiter segment of the mission, whereby Chapter 6 will provide an analysis on the specific orbital mechanics required to get to Europa. Chapter 7 will provide an analysis on the surface environment of Europa, Chapter 8 will provide further information on the Orbiter imaging system and design. Chapter 9 will provide information on the communications systems required to relay data back to Earth for analysis. Chapter 10 will detail information about the Lander segment of the mission. Chapter 11 will describe the Penetrator segment of the mission which will tunnel through the 2-10 km layer thick of ice on the surface. Chapter 12 will give further information on the specific instruments that will be taken onboard the penetrator for sampling, data collection, and further analysis. Finally, Chapter 13 will provide some preliminary analysis and verification of the mission and potential results, followed by a conclusion on the project and future work describing what would be needed to be further explored in order to guarantee mission success.