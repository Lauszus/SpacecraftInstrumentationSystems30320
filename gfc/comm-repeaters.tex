\autsubsection{Ice embedded Repeaters link}{Kristian Sloth Lauszus}

Another option for communicating between the penetrator and the lander would be to use multiple transceivers that would get deployed at certain intervals by the penetrator, as it is described by \citet{iceLink-scott}. Thus a series of repeaters was proposed, designed into 10 cm diameter cylinders with a height of 2-3 cm. Each repeater would have a 1 GHz patch antenna in the top and bottom of the cylinder and contain all the needed hardware for transmitting and receiving the data including a RTG for power and heat. Each repeater could also have temperature, pH, salinity, hydrophones and various other sensors embedded.

In the study the goal was to have a data rate of 10 kHz at 400 mW through a 10 km ice crust. In the paper the number of repeaters was found to be heavily influenced by salinity of the ice and temperature profile used. Thus the number of repeaters would increase in salty ice and if a layered ice temperature profile is used compared to a linear ice temperature model (see chapter \ref{sec:IceTemperatureProfile} and \ref{sec:IceTemperatureProfile2}). In the paper the salty ice is assumed to contain 13 PPM (Parts Per Million) salt impurities. For the linear ice temperature model the minimum amount of repeater was found to be 4 for a ice thickness of 3.5 pure ice, while 14 transceivers would be needed for 10 km salty ice. While the number of repeaters needed for the layered ice temperature profile would be vary from 8 for 5 km and 34 for 20 km pure and salty ice respectively.

However as \citet{iceLink-scott} also points out the distance between the receivers could be increased at the cost of a lower data rate, thus if the ice ends up being thicker or more salty than first expected, this could be compensated for by the on-board computer in the penetrator. On the other hand if the ice is more pure the distance could be increased or the data rate increased.

One way for the penetrator to deploy the repeaters would be to make the repeaters buoyant in such a way that they would float upward and align themselves in such a way that the antennas point up and down. A new repeater could then be deployed when the signal strength drops below a certain threshold. One problem with using the repeaters is that the complex dielectric permittivity increase in the order of 1000 times in water compared to ice, thus the penetrator might have to sit and wait for the repeater to freeze if a minimum data rate is needed.

Another option would be to use the repeaters at the outer ice crust where the ice is colder and the risk of fractures in the ice is higher, then once the penetrator gets below a certain depth a tether could be used to connect to the last repeater deployed at the warmer inner ice, where the risk of a tether breaking is less. This is especially useful if the layered ice temperature profile ends up being close to reality, as the temperature of the ice increases significantly in the beginning of the ice sheet, as discussed in chapter \ref{sec:IceTemperatureProfile2}. However then further studies would be needed in order to get more certainty regarding the ice temperature profile of Europa. Also further studies are needed regarding the salinity, as it is critical in order to estimate the number of repeaters needed.
