%   Include as introductory section in Communication Systems
%   target 0.5p
\autsection{Initial Considerations}{Gustavo Feijóo Carrillo}

In this chapter, the different links for each misssion stage are describe in full length. This section tries to elaborate a broad picture of the different scenarios as well as establishing initial requirements and assumptions to facilitate the task of designing the communication systems for the mission.

Three main links have been defined for the mission, earth-to-orbiter, orbiter-to-lander and lander-to-penetrator.

\paragraph{Earth to Orbiter}
As stated before, this type of link is at a higher readiness level since is the most common type of systems for earth orbiter missions, as well as previous interplanetary missions. Employing an X-band transponder for down/up-link will make the main channel for Telemetry and Command with the SC during all mission life. Additional to this, a Ka-band transmitter is used for high rate download of science data. Both transponders will have access to high gain- and medium gain antennas (HGA and MGA). The HGA will be fixed and the MGA will be provided with 2-axis steering capabilities to ensure a line of sight during maneuvers or at emergency situation minimize risk of losing communications.

\paragraph{Orbiter to Lander}
With the capabilities of current deep space antennas a communication link direct to earth (DTE) from the lander to ground control would be feasible but this systems is not studied further due to mass and bulk constraints for the lander module which will in-turn establish a relaying link between penetrator and orbiter, making the orbiter the last relaying block in the complete downlink chain.

\paragraph{Lander to Penetrator}
This is where a mission 'show stopper' may come up, and it has been a big effort to design and mitigate any possible risk to the performance of this link. Several types of link have been considered, from a tethered solution that would circumvent all the problems related to having ice as a means of propagation, to the wireless approach. It is assumed a maximum depth of 10km for this study as a goal for communication with the penetrator through the ice crust.

\paragraph{Different communication environments}
From launch until reaching the sub-ice ocean in Europa, the SC and its subsystems will be faced with different levels of radiation and temperature profiles, as well as noise from solar system bodies, milky way's center and cosmic origin. Regarding noise, at microwave frequencies used for the links, Jupiter is a blinding source of noise and all communications must be done in the anti-Jovian face of Europa to avoid this high noise levels.

