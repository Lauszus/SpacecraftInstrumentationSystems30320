\autsubsection{Tethered Link}{Kristian Sloth Lauszus}
One option for the penetrator to surface communication would be to use a tethered link connected from the end of the penetrator to the surface station. For instance a optical fiber could be used. The main benefit when using optical fiber for communication is the huge data rate of 10 Gb/s and beyond, which will allow the penetrator to broadcast scientific data, images and even video to the lander. Another advantage is that it is not unusual for optical fiber communication to work at up to 100 km without the need of any amplification\cite{website:optical_fiber_info}, which is way more than the worst case scenario of 36 km, as described in chapter \ref{sec:structural_profile}. Thus the optical fiber is very interesting, as it will basically allow one to send all the scientific data to the lander with little to any concerns about the data rate.

\subsubsection{Optical fiber strain}

However one major concern is whenever the strain from the ice caused by the tidal waves will exceed the capabilities of the fiber and eventually break it. This will be discussed in detail in the following section.\\

\noindent
According to Hook's law the force along the longitudinal axis of the optical fiber is given by:
\begin{equation}
	 F = k \, \Delta l
\end{equation}
Where $k$ is the elastic constant of the optical fiber and $\Delta l$ is the relative deformation caused by the perturbation force $F$. This law is fulfilled as long as the deformation does not exceed the elastic limit of the optical fiber which will prevent the fiber from retracting back to its original shape.

By knowing the Young's modulus of the optical fiber once can calculate the force of the perturbation using the following formula:
\begin{equation}
	F = E_G \, A \frac{\Delta l}{l}
\end{equation}
Where $E_G$ is the Young's modules, $A$ is the area of the optical fiber and $l$ is the length of the optical fiber.

By measuring the force applied to an optical fiber and plotting it versus the strain one can estimate the Young's modulus at the point where the slope of plot is no longer linear. Furthermore this point will also indicate the maximum strain of a given optical fiber. This strain number is very useful in our case, as this tell us how much the fiber can stretch before it is degraded. A typical number for protected optical fiber is about 3-5 percent\cite{article:optical_fiber_properties,article:optical_fiber_mechanical}. Since the column above the penetrator will start to freeze up again it is very important that the vertical strain of the ice does not exceed this critical value, as this will cause the optical fiber to break.

\subsubsection{Discussion}

Fortunately the vertical strain of the ice sheet on Europa is only estimated to be 0.3 \%, however this does not take into effect any potential cracks that occurs at the ice surface, as described in chapter \ref{sec:surface_studies}. This might cause the ice to shift horizontally, thus risking to break the cable. Thus the tethered link is very risky and should properly not be used as the only form of communication\cite{book:communication}.

The optical fiber could be wound up on a spool with a total weight of only a few kilograms\cite{book:communication}. However the wheel will also add up in complexity of the system and increase the volume. There will also be a risk of the wheel getting stuck while unwinding the optical fiber. Another option would be to use even more rigid cable, but this will increase volume and mass significantly\citet{iceLink-scott}.
