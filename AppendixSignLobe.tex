\autchapter{Half-wave Dipole Radiation Pattern}{Rasmus Lundby Pedersen}\label{app:rasmusmatlab}

This is the matlab code used for plotting the radiation pattern of a half-wave dipole placed above a large PEC plane.

\begin{lstlisting}
close all;
clear all;
clc;

format shorteng;

% Assuming r >> lambda

%For clarity, only has symbolic value
syms lambda
syms r
syms eta
syms I 
%%

phi = 0;
beta = 2*pi/lambda;

h = lambda*[0 1/4 1/2 3/4 2];  % The E-field will be calculated at 0, 1/4, 1/2, 3/4
% and 2 wavelengths above an infite PEC plane.
l = lambda/2;

theta = linspace(-pi,pi,1000);

E = zeros(4,length(theta));

%%
for k = 1:length(h)
    E = j*eta*(beta*I*l*1/(4*pi*r))*sqrt(1-sin(theta).^2.*sin(phi)^2).*(2*j
    .*sin(beta*h(k).*cos(theta))); 
    %Calculate the E-field for each of the k number of increment heights.
    E_norm = E/(beta*eta*l*I/(4*pi*r)); % Normalize
    
    % Plot label
    if k == 4
        ylabel('Relative field amplitude')
    end
    figure(1)
    subplot(length(h),1,k)
    plot(theta*180/pi,abs(E_norm))
    h1 = k/4-1/4;
    legend(['h = ' num2str(h1) '\lambda'])
    axis([-90. 90. 0 2.5])
end
xlabel('\Theta  (Degree \circ)')

\end{lstlisting}