\section{Analysis of the Imaging System}
The Europa Reconnaissance Imaging System (ERIS) plays an important role in the first phases of the mission. During the early stages, the imaging system will be used to map the surface of Europa and provide the necessary data for selecting a suitable landing site for the lander and penetrator. At this point, it is expected to have one or more cameras on the orbiter, mapping the surface of Europa in both low and high resolution. 
%\section{The Strawman Mission}
\subsection{The purpose of ERIS}
The Imaging System will be located in the orbiter, performing its primary objectives during the early stages of the life finding mission mission and performing the secondary objectives after a successful landing on Europa.

It is expected that the imaging system will be the main (and only) source of up-to-date, high resolution images throughout the mission. Therefore, it will be the only provider of the data that can be used for selecting a suitable landing site for the Europa Lander. It is assumed, that the current images provided from the Galileo Missions does not provide a high enough resolution to select a suitable landing site. This is the main reason why the relatively costly imaging system is required to perform additional mapping of the surface. 
\todo[inline]{Provide source for this assumption, i.e. the resolution is too low using the sources from Galileo. Image examples?}
\subsection{Landing Site Selection}
During the first phase of the mission, the Imaging System will map the surface. The purpose of the mapping is to provide enough information about the surface, to ensure a suitable landing site can be selected and to ensure the landing will be successful. When selecting the landing site, the following criteria will be considered:
\begin{description}
    \item[Roughness and Elevation] The roughness of the surface can be a great danger for the lander. Large boulders can be fatal, blocking the penetrator from touching the ice or causing misalignment. Boulders can also cause the lander to flip, after an otherwise successful landing. Surface elevation is also very dangerous for the mission - too steep, and the lander can topple over, ending the mission immediately. Therefore, it is necessary to investigate technologies, that can be used to map the surface in high resolution, ensuring large objects can be identified and avoided, making it possible to land on a relatively flat surface.
    
    While high resolution imaging will make it possible to identify dangerous objects, it may not be possible to create topographic data of the lunar surface, unless the cameras are operated in stereo. Therefore, these technologies will also be investigated.
    \item[Radiation] Some areas of the moon have a lower radiation, making them more suitable for the landing site. These areas were discussed previously, in section (\ref{sec:radiation_environment}).
    \item[Chemistry] The current knowledge about the surface chemistry of Europa is very limited. However, spectroscopic data from the Galileo NIMS spectrometer hints that sulphur-rich regions exists. To investigate the surface chemistry, multispectral imaging should be investigated further, as it can provide data about the surface composition of Europa. Knowledge about the surface chemistry is necessary, as some regions may be very unsuited for a landing site, due to the hazardous chemical compounds that are expected in these areas.
    \todo[inline]{Refer to the previous section about surface composition. }
    \item[Proximity to ocean] Naturally, the selected landing site should be close to the ocean, as it is expected that the ice thickness (and therefore mission duration) will vary, depending on the selected landing site.
    
    Ice sounding radars are normally used for assessing the ice structure but this analysis has focus on technologies, that can be used to enhance the imaging system. In addition to surface material assessment, recent papers suggest that spectroscopic data can also be used for analysing the ice surface\cite{naegeli2015a}.
    \item[Communication window] The communication window will be affected by the choice of landing site.
    \todo[inline]{Communication Window. more to add Ifikratis}
\end{description}
\subsection{Manual vs. Automatic selection}
Due to the limited bandwidth, it is not realistic to transmit all the high resolution images back to earth, before the lander has made it to the surface. Instead, a selective approach will used, where all low resolution images will be transmitted to earth but only the regions of interests (ROIs) will be transmitted in back in high resolution. It is expected that the high resolution imaging will require a large amount of on-board storage. A large part of the available communication link capacity will also be required. At this point, the specific requirements are not known but estimates will be calculated later in this report.

Depending on the processing power available for the imaging system, it could be possible to perform an autonomous selection of the landing site. However, due to the lack of knowledge about the lunar surface at this point, it will be very difficult to design a selection algorithm beforehand. Relying on a fully autonomous system is not commonly done in the industry [source]\todo[inline]{Remember source regarding landing autonomy} and creates an unnecessary risk. The alternative, a manual selection, is not difficult to implement, but requires two way communication to the spacecraft. The analysis will focus on providing the necessary data for a manual landing site selection but autonomous systems may still be useful to select and filter irrelevant images. The autonomous approach will be investigated briefly, as it can be a way to reduce the load on the communication and storage system.    \todo[inline]{Refer to Kristians section about image processing}

At this point, the following approach is planned for the landing site selection: First, the ROIs are identified manually, using the low resolution images provided by the wide angle camera (WAC). By transmitting the low resolution images down to earth, it is possible for the ground station to select the ROIs and instruct the spacecraft to transmit the high resolution imaging from the narrow angle camera, as soon as possible.
%instruct the spacecraft to investigate these areas further, using the narrow angle camera (NAC).
At this point, it is not planned to dynamically change the flyby trajectory to investigate some areas further. Instead, both the WAC and the NAC will be operating during each flyby but only the WAC images will be transmitted down to earth, whereas the NAC images will be stored until the ground station requests higher resolution images of specific zones. By using the selective approach, only the relevant data will be transmitted, avoiding excessive load of the communication link.

It is expected that minimum 30\%\footnote of the lunar surface must be mapped, using the low resolution. This should provide the scientists with enough data to select several regions of interests, ultimately selecting a suitable landing site. The remaining 70\% of the lunar surface is expected to be unsuited for the landing site, based on the  current knowledge about the surface.
\todo[inline]{Source regarding 30 pct coverage}
\subsection{Navigation - Orbit Determination}
As a secondary objective, the imaging system should provide data for orbit determination. Using the imaging system for both mapping and for tracking could be possible, but it may be difficult to do it at the same time. 

Orbit determination covers the task of continuously keeping track of the the position, where the spacecraft has been (orbit reconstruction), and where the spacecraft will be in the future (orbit prediction)\cite{doody2011spacefl} to ensure the spacecraft follows the required trajectory. Since the orbiter will be equipped with imaging instruments, they can be used to observe planets or satellites together with the star field. Often, the primary body such as Europa will be overexposed, ensuring the background star field will still be visible. This method provides the so called opnav images\cite{doody2011spacefl}. %The images will then be transmitted together with the rest of the telemetry data. 
The images provide accurate data about the trajectory of the spacecraft, aiding the orbit determination.

Being able to track the spacecraft and predict the movement is certainly required for an interplanetary mission. Predicting the location of the spacecraft also makes it possible for the deep space network (DSN) to follow the spacecraft, ensuring stable communication throughout the mission. The navigational data are essential for the on-board systems, as they provide the flight path control system with the necessary data to successfully bring the spacecraft back on course, if it is deviating from the planned trajectory. The spacecraft will be drifting away from the planned trajectory, due to different disturbances encountered during the flight. Over a long distance, small disturbances will add up over time, causing a significant drift for the spacecraft. It is also important to keep in mind that trajectory corrections will never be perfect. A small misalignment of the thrusters or a delayed cut-off of the engines will all contribute to the trajectory drifts. %Depending on the choice of surface mapping technique, the drifts may severely affect the resulting images. 

Orbit reconstruction is essential to make sense of the scientific data collected by the spacecraft - including, but not limited to the data produced by the image system. The timestamps for the collected data and images must be paired with reference data such as the orbit and orientation, to be able to perform a landing at a specific location or to use the data for scientific purposes. Most imaging techniques relies on very accurate reconstruction of the trajectory and orientation, to be able to stitch the images together.
\subsection{Geo-localisation and Lander Guidance}
As an extra objective, the Imaging System will monitor the lander and provide geolocation and guidance services for the landing team. It is expected that the lander will not land exactly where it was initially planned. While interplanetary navigation has improved, it is still not perfect\cite{nasa2012}. Therefore, it should be possible to use the imaging system to locate the lander, after a successful landing. Knowing the final landing site is essential to ensure a successful communication link and to supervise the mission from the ground.