\autchapter{Sphere of Influence}{Ifikratis Kamenidis}\label{app:sphere_of_infl}  
The sphere of influence defines the oblate-spheroid shaped area around a massive object that its gravity dominates. For a body with mass m around another body of mass M, the sphere of influence is given by the formula:

\begin{equation}
r_{SOI} \approxeq a\cdot (m/M)^{2/5}
\end{equation}

where a is the semimajor axis of the body's orbit.

In the case of Jupiter around the Sun, substituting with the semimajor axis of Jupiter along with its mass and the mass of the Sun we get 
\begin{equation}
r_{SOI},J=48.2\cdot 10^6 Km.     
\end{equation}

In the Jupiter system we calculate for Ganymede around Jupiter,
\begin{equation}
r_{SOI,Gan}=2.435\cdot 10^4 Km    
\end{equation}
while for Europa 
\begin{equation}
r_{SOI,Eur}=9.722\cdot 10^3 Km    
\end{equation}
 