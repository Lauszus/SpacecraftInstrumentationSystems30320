\chapter{Communication Systems}

%   Include as introductory section in Communication Systems
%   target 0.5p
%\autsection{Initial Considerations}

In this chapter, the different links for each misssion stage are describe in full length. This section tries to elaborate a broad picture of the different scenarios as well as establishing initial requirements and assumptions to facilitate the task of designing the communication systems for the mission.

Three main links have been defined for the mission, earth-to-orbiter, orbiter-to-lander and lander-to-penetrator.

\paragraph{Earth to Orbiter}
As stated before, this type of link is at a higher readiness level since is the most common type of systems for earth orbiter missions, as well as previous interplanetary missions. Employing an X-band transponder for down/up-link will make the main channel for Telemetry and Command with the SC during all mission life, as well as accurate tracking of the spacecraft required for orbital maneuvers. Additional to this, a Ka-band transmitter is used for high data rate download of science data. Both transponders will have access to high gain- and medium gain antennas (HGA and MGA). The HGA will be fixed and the MGA will be provided with 2-axis steering capabilities to ensure a line of sight during maneuvers or at emergency situation minimize risk of losing communications.

Additionally, this link is necessary for tracking of the spacecraft, providing range measurements (with a resolution of $1m$) as well as angle and radial velocity which in conjunction allow for accurate positioning of a deep space SC necessary for trajectory correction and orbital maneuvers with higher efficiency and precision at interplanetary distances.

In the following subsection ESA's deep space communication network is presented, as well as a view of next generation very high data rate system, based on laser communication which could allow for deep space remote sensing missions (heavy on data loads).

\paragraph{Orbiter to Lander}
With the capabilities of current deep space antennas a communication link direct to earth (DTE) from the lander to ground control would be feasible but this systems is not studied further due to mass and bulk constraints for the lander module which will in-turn establish a relaying link between the penetrator vehicle and the orbiter. 

\paragraph{Lander to Penetrator}
This is where a mission 'show stopper' may come up, and it has been a big effort to design and mitigate any possible risk to the performance of this link. Several types of link have been considered, from a tethered solution that would circumvent all the problems related to having ice as a means of propagation, to the wireless approach. It is assumed a maximum depth of 10km for this study as a goal for communication with the penetrator through the ice crust.

\paragraph{Different communication environments}
From launch until reaching the sub-ice ocean in Europa, the SC and its subsystems will be faced with different levels of radiation and temperature profiles, as well as noise from solar system bodies, milky way's center and cosmic origin. Regarding noise, at microwave frequencies used for the links, Jupiter is a blinding source of noise and all communications must be done in the anti-Jovian face of Europa to avoid this high noise levels.



\section{Earth-to-Orbiter}

\autsubsection{ESA's Deep Space Communication Network}{Gustavo Feijóo Carrillo}

The International Telecommunications Union (ITU) defines Deep Space (DS) to start at a distance of 2 million km from the Earth's surface. Allocated frequency bands for DS operation according to ITU are:

\begin{itemize}
    \item S-Band: 2110-2120 MHz uplink, 2290-2300MHz downlink.
    \item X-Band: 7145-7190 MHz uplink, 8400-8450MHz downlink.
    \item Ka-Band: 34.2-34.7 GHz uplink, 31.8-32.3GHz downlink.
\end{itemize}

\noindent
ESA's effor to carry out interplanetary missions like Rosetta or Kepler Observatory as well as Icy Moon Exploration (JUICE). The capability of supporting present and future DS missions is the consequence of well established plan to expand the network of $15m$ tracking antennas into the deep space distances. The plan consisting on deploying three $35m$ DS antennas over the world, ensuring around the clock coverage to all interplanetary missions, which has been recently completed. These three DS antennas, DSA1 located in New Norcia (Australia), DSA2 in Cebreros (Spain) and DSA3 in Malargue (Argentina), are in operation since 2002, 2005 and 2013 respectively.

The DSA1-3 network will provide telecommand and telemetry links for a Europa mission but as important as the data links, is the tracking of the SC since the DSA-2 is equipped with Delta DOR (Delta Differential One-Way Ranging) capability, a new technology enabling highly precise spacecraft location and tracking.

\begin{figure}[htb]
	\centering
	\includegraphics[width=\textwidth]{figures/comms/ESTRACK-map}
	\caption{Map of ESA's ESTRACK network, allowing for deep space communication and tracking of the spacecraft.}
	\label{fig:ESTRACK-map}
\end{figure}

\paragraph{Description}
The telecommunication system will use redundant X and Ka transponders for telemetry reception and transmission. The amplifiers will be based on redundant $65W_{RF}$, travelling wave tube amplifiers for Ka-band, and $75W_{RF}$ for X-band, respectively. The downlink of the science telemetry would be in either X-band, or Ka-band, or with both systems operating simultaneously, in the case of making up for lost transmission windows in order to meet the baseline data volume. The high gain antenna will be fixed with a diameter of $3.2m$. A medium gain antenna would be based on a horn antenna with an opening angle of the $20^o$,which covers the maximum angular distance of the Earth from the Sun, when seen from Jupiter, and would therefore allow the MGA to be simply Sun-pointed during safe mode.

A two-axis steerable medium gain antenna (MGA) is to be provided to allow for communications during the path of the inner Solar System to perform gravity assist maneuvers around Venus requiring the HGA to be used as a thermal shield for Sun as well as Venus albedo. Furthermore, for distances $>2AU$ during the interplanetary phase, and during the Jupiter phase, the MGA would be used for Earth search during safe mode recovery.

\begin{figure}[htb]
	\centering
	\subfloat[DSA-2 at Cebreros-Spain]{
		\includegraphics[width=.48\textwidth]{figures/comms/DSA2}
		\label{fig:DSA2}
	}
	\subfloat[HGA on an orbiter SC]{
		\includegraphics[width=.48\textwidth]{figures/comms/orbiter-HGA}
		\label{fig:orbiter-HGA}
	}
	\caption{Picture of $35m$ DSA and depiction of fixed HGA for DS communication and tracking of the spacecraft.}
	\label{fig:DS-a}
\end{figure}

\newpage
\subsubsection{X- and Ka-band Link Budget}

\begin{figure}[htb]
	\centering
	\subfloat[X-band link budget]{
		\includegraphics[trim={4cm 4cm 3cm 2cm},clip,width=.48\textwidth]{figures/comms/linkBudget-Xband}
		\label{fig:budget-X}
	}
	\subfloat[Ka-band link budget]{
		\includegraphics[trim={4cm 4cm 3cm 2cm},clip,width=.48\textwidth]{figures/comms/linkBudget-Kband}
		\label{fig:budget-K}
	}
	\caption{Link budget for X- and Ka-band}
\end{figure}


\autsubsection{Wireless Laser communications}{Bhaaeddin Alhomsi}

Laser communications systems are wireless connections through the atmosphere. They work similarly to fiber optic links, except the beam is transmitted through free space. While the transmitter and receiver must require line-of-sight conditions, they have the benefit of eliminating the need for broadcast rights and buried cables.
 Laser communications systems can be easily deployed since they are inexpensive, small, low power and do not require any radio interference studies. The carrier used for the transmission signal is typically generated by a laser diode. Two parallel beams are needed, one for transmission and one for reception. 
Lasers have been considered for space communications since their realization in 1960. Specific advancements were needed in component performance and system engineering particularly for space qualified hardware. Advances in system architecture, data formatting and component technology over the past three decades have made laser communications in space not only viable but also an attractive approach into inter satellite link applications.
Optical systems have the advantage of extremely broad, unregulated bandwidth compared to RF systems. At Ka-band frequencies of 32 or 37-38 GHz, bandwidth is typically 500 MHz. For optical systems at 1.55 µm, the bandwidth may be 1000 times larger, allowing optical systems to carry substantially more information. For RF systems to compete in a bandwidth-constrained environments they must resort to bandwidth-efficient modulation for data rates above 1 Gbps, which is neither power- nor mass-efficient for the transmitting terminal. Optical systems typically would have smaller receive apertures and lower power efficient transmitters and receivers. While RF systems do suffer from atmospheric attenuation due to weather, especially at Ka-band and above, they have the capability of penetrating cloud cover, whereas optical systems do not.
The first efforts in space-based laser communications, achieved by Japan and Europe, showed some success in overcoming these hurdles. Japan's 1-Mb/s laser link to ground from the ETS-VI satellite in GEO in 1994—the first successful demonstration—was followed in 2001 by ESA’s SILEX/Artemis link demonstrations from GEO to ground and from GEO to low-Earth orbit (LEO). These initial experiments successfully demonstrated pointing, acquisition and tracking of narrow laser beams between spacecraft and directly to Earth stations, laying the groundwork for future systems in both Europe and Japan.
Development of FSOC flight systems continued in the early 2000 s. The U.S. government launched the GEOLite laser communications mission in 2001. In 2008, the German Aerospace Center demonstrated a data rate of 5.6 Gb/s across 4,000 km crosslinks in space between its TerraSAR-X satellite and a corresponding terminal on the NFIRE spacecraft managed by the U.S. Department of Defense. Europe is now building on that experience to provide up to 1.8 Gb/s of laser-driven bandwidth to its Earth-observing Sentinel satellites in LEO, which will be the first operational laser communication users of the European Data Relay Satellite (EDRS) system, launching into GEO in 2016.
The U.S. space agency has followed a more tentative path for laser communications in space. Although NASA initiated multiple efforts during the 1980s and 1990s, all were eventually cancelled due to growth in costs and the difficulty of obtaining reliable photonic components for use in space. But the economics of space laser communications changed significantly with the growth of terrestrial optical-fiber communications in the early 2000s, which suddenly boosted the availability of high-performance, low-cost components such as stable and efficient distributed-feedback (DFB) lasers, low-loss LiNbO3 modulators, and high-power and low-noise erbium-doped fiber amplifiers (EDFAs), all in the 1550-nm wavelength band. And the stringent environmental and reliability requirements of the Telcordia certifications, which govern terrestrial optical-communications equipment, are well-aligned with those for spaceflight.

\subsubsection{From moon to Earth: NASA’s LLCD mission}

NASA’s approach has been to leverage this Earth-based development for space, purchasing commercial components and, via rigorous space-qualification testing, moving them into new, reliable and lower-cost laser communications systems for both deep space and near Earth. Using that approach, NASA demonstrated its first laser communication system in space in 2013, with the Lunar Laser Communications Demonstration (LLCD) mission, aboard the Lunar Atmosphere Dust and Environment Explorer (LADEE). The mission broke new ground in a number of areas:
Longest-range dedicated optical communications link. LLCD demonstrated error-free data downlink rates of up to 622 Mb/s from the moon at a distance of some 400,000 km—ten times the range of earlier GEO-to-ground experiments, and thus overcoming a link loss that is 100 times greater. This included error-free operation through the turbulent atmosphere.
High data rates. LLCD was an order of magnitude higher in data rate than the best Ka-band radio system flown to the moon (100 Mb/s) on the 2009 Lunar Reconnaissance Orbiter.
High-definition video link. LLCD also demonstrated a 20-Mb/s uplink, which was used to transmit error-free high-definition video to and from the moon, a communication capability crucial to possible efforts to send humans beyond low-Earth orbit.
Pinpoint ranging. LLCD’s communication system provided simultaneous centimeter-class precision ranging to the spacecraft, which can be used to improve both spacecraft navigation and the gravity models of planetary bodies for science.
Low size, weight and power. LLCD’s space-based laser terminal required only half the mass (30.7 kg) and 25 \% less power (90 W) than the Lunar Reconnaissance Orbiter RF system (61 kg and 120 W, respectively).

The LLCD mission’s real breakthrough, however, was its demonstration that such a system could return real, high-value science data from the LADEE's instruments as they probed the moon. The LLCD space terminal and primary ground terminal—both designed, built and operated by the Massachusetts Institute of Technology (MIT) Lincoln Laboratory—showed near-instantaneous laser link acquisition on every possible pass, followed by closed-loop tracking of the 15-microradian uplink and downlink beams. Data was imparted with pulse-position modulation (PPM) of an amplified single-frequency laser, then transmitted across the vast distance to the ground receiver through narrowband spectral filtering, in front of a state-of-the-art photon-counting detector. This device consisted of 16 superconducting nanowire detector arrays (SNDAs), and is so sensitive that only two received photons were required for every error-free bit detected.

\subsubsection{Performance Advantages over RF}

\begin{enumerate}
	\item Cost consideration limits the aperture diameters to be much smaller than that of the RF system (0.3 m vs. 1 .5m for spacecraft antenna, and 1 Om vs. 70m for ground station). 
	\item Diode-pumped solid state laser has much lower power efficiency compared to RF amplifiers (10 \% vs 40 \%). EDFA technology can potentially achieve a better efficiency (20 \%). However, reduction in antenna gain and receiver sensitivity more than compensate for the increased efficiency. 
	\item Optical system is much more sensitive to pointing loss and atmospheric attenuation.
\end{enumerate}

\begin{figure}[htb]
\begin{center}
\includegraphics[width=1\columnwidth]{figures/laser-communication/bh3.jpg}
\caption{Comparison of optical and ka-band system performance}
\end{center}
\end{figure}
\noindent
Shown in Table 1 is the performance comparison between the proposed optical link and a near-term achievable RF link performance using Ka-band. Assuming equal power for the receiver and for monitor and control functions, the comparison is based on a constant DC power consumption by the transmit power amplifier. The optical link estimate is based on a 30 cmaperture diameter transmitter and a 1O m diameter receiver using 256-ary PPM and a I .06 jtm diode-pumped solid state laser. The Ka-band performance projection is based on the assumption that continuing improvements in Ka-band will lead to 
(a) implementation of Ka-band reception capability on the 70 m stations,
 (b) improved receiver aperture efficiency with either
the array feed or adjustable mirror technology, 
(c) improved spacecraft Ka-band transmitter power efficiency with high efficiency TWTAs, 
(d) Improved transmit aperture efficiency using off axis or displaced-axis antenna.

\subsubsection{Weather}

Despite of many such technological development, the major limitation of free-space laser communication (lasercom) performance is the atmosphere.  Atmospheric condition ultimately determines the laser communications systems pe for uplink-downlink , because a portion of the atmospheric path always includes turbulence and multiple scattering effects.
Heavy fog is a major weather constraint that almost completely blocks sources of light. Thus, it threatens FSO (Free-space optical communication) links by attenuating the light signal and almost breaking it.
 Light can be absorbed into the atmosphere. The phenomenon of absorption is caused mainly by the presence of water vapor and carbon dioxide in the air. These gases in addition to other types create what is called transmission windows that limit the passage of some light frequencies. The frequencies of laser sources are not in the range of absorption by these transmission windows, resulting in no effect on the communications link. laser sources are not affected by the atmospheric absorption of light during the transmission and receiving process.
Scattering is more of a concern to FSO links than absorption. This is true because scattering is a function of the light wavelength and the quantity and size of scattering elements in the atmosphere. scatter FSO light sources but the effect is considered negligible. The main weather contributor to signal attenuation is fog. Fog occurs when humidity of the air reaches a certain saturation level which condensates vapor to water droplets of microns of radius. These droplets are the major cause of scattering for infrared light. Even though fog droplets are smaller on average than cloud droplets but are huge when compared to the wavelength of the light.
The method of transmitting data from JIMO back to Earth is to use a free-space laser communications link from JIMO to an Earth-orbiting relay satellite. Using an optical receiver on an Earth-orbiting relay satellite is advantageous because it makes it possible to overcome the severe  atmospheric losses that may be experienced in direct optical transmission to a ground-based receiver.

\subsubsection{Proposed laser communication between Europa and earth}

The method of transmitting data from JIMO back to Earth is to use a free-space laser communications link from JIMO to an Earth-orbiting relay satellite. Using an optical receiver on an Earth-orbiting relay satellite is advantageous because it makes it possible to overcome the severe  atmospheric losses that may be experienced in direct optical transmission to a ground-based receiver. High data transmission rates can be achieved by using Wavelength Division Multiplexing (WDM) with multiple optical carriers each at different wavelengths. Additionally, a single transmitting telescope on JIMO can support the transmission of an optical beam that optically combines or multiplexes the data-modulated outputs from a multiple number of laser transmitters, each operating at different wavelengths.

Transmit optical aperture diameters ranging from 30 cm to 90 cm are evaluated. A single receiving telescope on the Earth-orbiting relay satellite is required with an optical aperture size that must be large enough to collect a sufficient amount of propagated light from the arriving optical light beam for carrying out reliable data demodulation and decoding. Receive optical apertures of 2.4 m and 3.6 m.

\subsubsection{Wavelength}

Selection of wavelengths for optical communications depends on an understanding of the propagation channel, both through free space and atmosphere, and on the availability of components and subsystems including lasers, detectors, and optics. Additionally, issues pertaining to availability and reliability of components, especially lasers and detectors for user spacecraft, are critical to the selection of a viable communications wavelength. Considerations of missions and operational issues can also profoundly affect the choice of wavelength. Free space propagation loss decreases with wavelength, and provides the single most compelling reason to choose shorter wavelengths for laser communications. The angular beam diameter for a diffraction limited beam as measured by the first Airy disk of the diffraction pattern for a circular aperture is given by 2.44[\lambda/D], where D is the diameter of the transmitting aperture and \lambda is the wave length. Energy density at the receiver is inversely proportional to the square of the beam diameter. For a given distance z and transmitting aperture, the received energy density increases as [1/\lambda^2] below Fig. ,shows that the energy density decreased by three orders of magnitude as the wavelength increases from 0.4 to 12.5 jim. This provides a strong argument to choose shorter wavelengths for laser communications.
\begin{figure}[htb]
\begin{center}
\includegraphics[width=0.7\columnwidth]{figures/laser-communication/bh4.jpg}
\caption{wavelength and power density}
\end{center}
\end{figure}
\\
We propose to use wavelengths on the ITU (International Telecommunication Union) grid specified for terrestrial fiber networks in the C band with nominal wavelengths between 1.53 and 1.57 microns (5 THz bandwidth). This choice of wavelengths takes advantage of the extensive terrestrial fiber network WDM technology currently available, including laser sources, photodiode detectors, erbium-doped fiber amplifiers (EDFA) for optical amplification, and WDM multiplexers and demultiplexers. 

\subsubsection{Modulation Type}

The large geometric space loss for communicating with deep space optical missions such as JIMO requires power efficient communication techniques.
optical detection receiver, is proposed to provide a highly power-efficient high data transmission rate system with reasonable implementation complexity. Specifically, a 256-slot PPM modulation scheme with RZ pulse shaping is considered, where each symbol carries 8 bits, requiring a 32-fold bandwidth expansion relative to the conventional binary on/off keying modulation normally employed in terrestrial fiber networks.

\subsubsection{Laser Source}

A high-power laser source at each wavelength can be implemented using a conventional laser diode followed by an EDFA power amplifier to produce 5 watts of launch power. Link closure at a 1.55-micron wavelength is then achieved for a 2.4 m receive aperture at data rates ranging from 2.5 Mbps for a 30 cm transmit aperture to 25 Mbps for a 90 cm transmit aperture. Increasing the receive aperture to 3.6 m increases the corresponding data transmission rates.

\subsubsection{Tx and Rx Aperture}

A single receiving telescope on the Earth-orbiting relay satellite is required with an optical aperture size that must be large enough to collect a sufficient amount of propagated light ,As we can see, the total telemetry data rate increased by increasing the transmitter aperture and receiver aperture, and adding more laser diode will increase the data rate. 

\begin{figure}[htb]
\begin{center}
\includegraphics[width=0.7\columnwidth]{figures/laser-communication/bh5.png}
\caption{Achievable Data Return for laser Communication System}
\end{center}
\end{figure}

\subsubsection{Pointing system}

Due to high data rates and reliability, the stability of the laser beam pointing is still a key technique which needs to be solved; otherwise, the beam pointing jitter noise would reduce the communication quality or, even worse, would make the inter-satellite laser communication impossible. 

\subsubsection{Weight, size, and power requirements}

The weight, size, and power requirements for a JIMO transmitter employing two wavelengths are estimated using extrapolation from previous work performed at The Aerospace Corporation.
\begin{figure}[htb]
\begin{center}
\includegraphics[width=1\columnwidth]{figures/laser-communication/bh11.jpg}
\caption{weight, size, and power requirements}
\end{center}
\end{figure}

\subsubsection{Mission and Coverage limitation}

For almost all deep space missions, the mission profile will impose limits on spacecraft attitude and pointing of spacecraft during certain mission phases. Examples of such mission phases are the launch phase or inner cruise phase where spacecraft attitude is constrained by the trajectory or thermal consideration.there will be coverage holes that can potentially limit the mission planning.
-Difficulty in maintaining link with degraded spacecraft performance: This can include degraded station-keeping capability, degraded star tracker performance, or loss of time reference, will lead to unmanageable acquisition time.

The detection sensitivity is significantly worse at optical frequency, even with the use of high order PPM modulation . The optical receiver sensitivity can further degrade to 20-30 photons/bit under daytime conditions with current receiver technology.

Due to a high distance and difficulty of continuous acquisition and tracking, we can not use laser communication for this mission.


%This is surface-to-orbiter subsection
\section{Earth-to-Orbiter} % Rasmus

1. Introduction and considerations for the link
   1. General Considerations for a Deep Space Mission
   2. HG Link
   3. MG Link
   4. LG Link
   5. Telemetry Uplink/Downlink
   6. Data Downlink
   7. Safe Mode
1. Link Drivers
   1. Dataload
   2. Spacecraft Attitude
1. Link Budget
2. Solution Proposal
3. Drivers to other systems

\autsubsection{Wireless Laser communications}{Bhaaeddin Alhomsi}

Laser communications systems are wireless connections through the atmosphere. They work similarly to fiber optic links, except the beam is transmitted through free space. While the transmitter and receiver must require line-of-sight conditions, they have the benefit of eliminating the need for broadcast rights and buried cables.
 Laser communications systems can be easily deployed since they are inexpensive, small, low power and do not require any radio interference studies. The carrier used for the transmission signal is typically generated by a laser diode. Two parallel beams are needed, one for transmission and one for reception. 
Lasers have been considered for space communications since their realization in 1960. Specific advancements were needed in component performance and system engineering particularly for space qualified hardware. Advances in system architecture, data formatting and component technology over the past three decades have made laser communications in space not only viable but also an attractive approach into inter satellite link applications.
Optical systems have the advantage of extremely broad, unregulated bandwidth compared to RF systems. At Ka-band frequencies of 32 or 37-38 GHz, bandwidth is typically 500 MHz. For optical systems at 1.55 µm, the bandwidth may be 1000 times larger, allowing optical systems to carry substantially more information. For RF systems to compete in a bandwidth-constrained environments they must resort to bandwidth-efficient modulation for data rates above 1 Gbps, which is neither power- nor mass-efficient for the transmitting terminal. Optical systems typically would have smaller receive apertures and lower power efficient transmitters and receivers. While RF systems do suffer from atmospheric attenuation due to weather, especially at Ka-band and above, they have the capability of penetrating cloud cover, whereas optical systems do not.
The first efforts in space-based laser communications, achieved by Japan and Europe, showed some success in overcoming these hurdles. Japan's 1-Mb/s laser link to ground from the ETS-VI satellite in GEO in 1994—the first successful demonstration—was followed in 2001 by ESA’s SILEX/Artemis link demonstrations from GEO to ground and from GEO to low-Earth orbit (LEO). These initial experiments successfully demonstrated pointing, acquisition and tracking of narrow laser beams between spacecraft and directly to Earth stations, laying the groundwork for future systems in both Europe and Japan.
Development of FSOC flight systems continued in the early 2000 s. The U.S. government launched the GEOLite laser communications mission in 2001. In 2008, the German Aerospace Center demonstrated a data rate of 5.6 Gb/s across 4,000 km crosslinks in space between its TerraSAR-X satellite and a corresponding terminal on the NFIRE spacecraft managed by the U.S. Department of Defense. Europe is now building on that experience to provide up to 1.8 Gb/s of laser-driven bandwidth to its Earth-observing Sentinel satellites in LEO, which will be the first operational laser communication users of the European Data Relay Satellite (EDRS) system, launching into GEO in 2016.
The U.S. space agency has followed a more tentative path for laser communications in space. Although NASA initiated multiple efforts during the 1980s and 1990s, all were eventually cancelled due to growth in costs and the difficulty of obtaining reliable photonic components for use in space. But the economics of space laser communications changed significantly with the growth of terrestrial optical-fiber communications in the early 2000s, which suddenly boosted the availability of high-performance, low-cost components such as stable and efficient distributed-feedback (DFB) lasers, low-loss LiNbO3 modulators, and high-power and low-noise erbium-doped fiber amplifiers (EDFAs), all in the 1550-nm wavelength band. And the stringent environmental and reliability requirements of the Telcordia certifications, which govern terrestrial optical-communications equipment, are well-aligned with those for spaceflight.

\subsubsection{From moon to Earth: NASA’s LLCD mission}

NASA’s approach has been to leverage this Earth-based development for space, purchasing commercial components and, via rigorous space-qualification testing, moving them into new, reliable and lower-cost laser communications systems for both deep space and near Earth. Using that approach, NASA demonstrated its first laser communication system in space in 2013, with the Lunar Laser Communications Demonstration (LLCD) mission, aboard the Lunar Atmosphere Dust and Environment Explorer (LADEE). The mission broke new ground in a number of areas:
Longest-range dedicated optical communications link. LLCD demonstrated error-free data downlink rates of up to 622 Mb/s from the moon at a distance of some 400,000 km—ten times the range of earlier GEO-to-ground experiments, and thus overcoming a link loss that is 100 times greater. This included error-free operation through the turbulent atmosphere.
High data rates. LLCD was an order of magnitude higher in data rate than the best Ka-band radio system flown to the moon (100 Mb/s) on the 2009 Lunar Reconnaissance Orbiter.
High-definition video link. LLCD also demonstrated a 20-Mb/s uplink, which was used to transmit error-free high-definition video to and from the moon, a communication capability crucial to possible efforts to send humans beyond low-Earth orbit.
Pinpoint ranging. LLCD’s communication system provided simultaneous centimeter-class precision ranging to the spacecraft, which can be used to improve both spacecraft navigation and the gravity models of planetary bodies for science.
Low size, weight and power. LLCD’s space-based laser terminal required only half the mass (30.7 kg) and 25 \% less power (90 W) than the Lunar Reconnaissance Orbiter RF system (61 kg and 120 W, respectively).

The LLCD mission’s real breakthrough, however, was its demonstration that such a system could return real, high-value science data from the LADEE's instruments as they probed the moon. The LLCD space terminal and primary ground terminal—both designed, built and operated by the Massachusetts Institute of Technology (MIT) Lincoln Laboratory—showed near-instantaneous laser link acquisition on every possible pass, followed by closed-loop tracking of the 15-microradian uplink and downlink beams. Data was imparted with pulse-position modulation (PPM) of an amplified single-frequency laser, then transmitted across the vast distance to the ground receiver through narrowband spectral filtering, in front of a state-of-the-art photon-counting detector. This device consisted of 16 superconducting nanowire detector arrays (SNDAs), and is so sensitive that only two received photons were required for every error-free bit detected.

\subsubsection{Performance Advantages over RF}

\begin{enumerate}
	\item Cost consideration limits the aperture diameters to be much smaller than that of the RF system (0.3 m vs. 1 .5m for spacecraft antenna, and 1 Om vs. 70m for ground station). 
	\item Diode-pumped solid state laser has much lower power efficiency compared to RF amplifiers (10 \% vs 40 \%). EDFA technology can potentially achieve a better efficiency (20 \%). However, reduction in antenna gain and receiver sensitivity more than compensate for the increased efficiency. 
	\item Optical system is much more sensitive to pointing loss and atmospheric attenuation.
\end{enumerate}

\begin{figure}[htb]
\begin{center}
\includegraphics[width=1\columnwidth]{figures/laser-communication/bh3.jpg}
\caption{Comparison of optical and ka-band system performance}
\end{center}
\end{figure}
\noindent
Shown in Table 1 is the performance comparison between the proposed optical link and a near-term achievable RF link performance using Ka-band. Assuming equal power for the receiver and for monitor and control functions, the comparison is based on a constant DC power consumption by the transmit power amplifier. The optical link estimate is based on a 30 cmaperture diameter transmitter and a 1O m diameter receiver using 256-ary PPM and a I .06 jtm diode-pumped solid state laser. The Ka-band performance projection is based on the assumption that continuing improvements in Ka-band will lead to 
(a) implementation of Ka-band reception capability on the 70 m stations,
 (b) improved receiver aperture efficiency with either
the array feed or adjustable mirror technology, 
(c) improved spacecraft Ka-band transmitter power efficiency with high efficiency TWTAs, 
(d) Improved transmit aperture efficiency using off axis or displaced-axis antenna.

\subsubsection{Weather}

Despite of many such technological development, the major limitation of free-space laser communication (lasercom) performance is the atmosphere.  Atmospheric condition ultimately determines the laser communications systems pe for uplink-downlink , because a portion of the atmospheric path always includes turbulence and multiple scattering effects.
Heavy fog is a major weather constraint that almost completely blocks sources of light. Thus, it threatens FSO (Free-space optical communication) links by attenuating the light signal and almost breaking it.
 Light can be absorbed into the atmosphere. The phenomenon of absorption is caused mainly by the presence of water vapor and carbon dioxide in the air. These gases in addition to other types create what is called transmission windows that limit the passage of some light frequencies. The frequencies of laser sources are not in the range of absorption by these transmission windows, resulting in no effect on the communications link. laser sources are not affected by the atmospheric absorption of light during the transmission and receiving process.
Scattering is more of a concern to FSO links than absorption. This is true because scattering is a function of the light wavelength and the quantity and size of scattering elements in the atmosphere. scatter FSO light sources but the effect is considered negligible. The main weather contributor to signal attenuation is fog. Fog occurs when humidity of the air reaches a certain saturation level which condensates vapor to water droplets of microns of radius. These droplets are the major cause of scattering for infrared light. Even though fog droplets are smaller on average than cloud droplets but are huge when compared to the wavelength of the light.
The method of transmitting data from JIMO back to Earth is to use a free-space laser communications link from JIMO to an Earth-orbiting relay satellite. Using an optical receiver on an Earth-orbiting relay satellite is advantageous because it makes it possible to overcome the severe  atmospheric losses that may be experienced in direct optical transmission to a ground-based receiver.

\subsubsection{Proposed laser communication between Europa and earth}

The method of transmitting data from JIMO back to Earth is to use a free-space laser communications link from JIMO to an Earth-orbiting relay satellite. Using an optical receiver on an Earth-orbiting relay satellite is advantageous because it makes it possible to overcome the severe  atmospheric losses that may be experienced in direct optical transmission to a ground-based receiver. High data transmission rates can be achieved by using Wavelength Division Multiplexing (WDM) with multiple optical carriers each at different wavelengths. Additionally, a single transmitting telescope on JIMO can support the transmission of an optical beam that optically combines or multiplexes the data-modulated outputs from a multiple number of laser transmitters, each operating at different wavelengths.

Transmit optical aperture diameters ranging from 30 cm to 90 cm are evaluated. A single receiving telescope on the Earth-orbiting relay satellite is required with an optical aperture size that must be large enough to collect a sufficient amount of propagated light from the arriving optical light beam for carrying out reliable data demodulation and decoding. Receive optical apertures of 2.4 m and 3.6 m.

\subsubsection{Wavelength}

Selection of wavelengths for optical communications depends on an understanding of the propagation channel, both through free space and atmosphere, and on the availability of components and subsystems including lasers, detectors, and optics. Additionally, issues pertaining to availability and reliability of components, especially lasers and detectors for user spacecraft, are critical to the selection of a viable communications wavelength. Considerations of missions and operational issues can also profoundly affect the choice of wavelength. Free space propagation loss decreases with wavelength, and provides the single most compelling reason to choose shorter wavelengths for laser communications. The angular beam diameter for a diffraction limited beam as measured by the first Airy disk of the diffraction pattern for a circular aperture is given by 2.44[\lambda/D], where D is the diameter of the transmitting aperture and \lambda is the wave length. Energy density at the receiver is inversely proportional to the square of the beam diameter. For a given distance z and transmitting aperture, the received energy density increases as [1/\lambda^2] below Fig. ,shows that the energy density decreased by three orders of magnitude as the wavelength increases from 0.4 to 12.5 jim. This provides a strong argument to choose shorter wavelengths for laser communications.
\begin{figure}[htb]
\begin{center}
\includegraphics[width=0.7\columnwidth]{figures/laser-communication/bh4.jpg}
\caption{wavelength and power density}
\end{center}
\end{figure}
\\
We propose to use wavelengths on the ITU (International Telecommunication Union) grid specified for terrestrial fiber networks in the C band with nominal wavelengths between 1.53 and 1.57 microns (5 THz bandwidth). This choice of wavelengths takes advantage of the extensive terrestrial fiber network WDM technology currently available, including laser sources, photodiode detectors, erbium-doped fiber amplifiers (EDFA) for optical amplification, and WDM multiplexers and demultiplexers. 

\subsubsection{Modulation Type}

The large geometric space loss for communicating with deep space optical missions such as JIMO requires power efficient communication techniques.
optical detection receiver, is proposed to provide a highly power-efficient high data transmission rate system with reasonable implementation complexity. Specifically, a 256-slot PPM modulation scheme with RZ pulse shaping is considered, where each symbol carries 8 bits, requiring a 32-fold bandwidth expansion relative to the conventional binary on/off keying modulation normally employed in terrestrial fiber networks.

\subsubsection{Laser Source}

A high-power laser source at each wavelength can be implemented using a conventional laser diode followed by an EDFA power amplifier to produce 5 watts of launch power. Link closure at a 1.55-micron wavelength is then achieved for a 2.4 m receive aperture at data rates ranging from 2.5 Mbps for a 30 cm transmit aperture to 25 Mbps for a 90 cm transmit aperture. Increasing the receive aperture to 3.6 m increases the corresponding data transmission rates.

\subsubsection{Tx and Rx Aperture}

A single receiving telescope on the Earth-orbiting relay satellite is required with an optical aperture size that must be large enough to collect a sufficient amount of propagated light ,As we can see, the total telemetry data rate increased by increasing the transmitter aperture and receiver aperture, and adding more laser diode will increase the data rate. 

\begin{figure}[htb]
\begin{center}
\includegraphics[width=0.7\columnwidth]{figures/laser-communication/bh5.png}
\caption{Achievable Data Return for laser Communication System}
\end{center}
\end{figure}

\subsubsection{Pointing system}

Due to high data rates and reliability, the stability of the laser beam pointing is still a key technique which needs to be solved; otherwise, the beam pointing jitter noise would reduce the communication quality or, even worse, would make the inter-satellite laser communication impossible. 

\subsubsection{Weight, size, and power requirements}

The weight, size, and power requirements for a JIMO transmitter employing two wavelengths are estimated using extrapolation from previous work performed at The Aerospace Corporation.
\begin{figure}[htb]
\begin{center}
\includegraphics[width=1\columnwidth]{figures/laser-communication/bh11.jpg}
\caption{weight, size, and power requirements}
\end{center}
\end{figure}

\subsubsection{Mission and Coverage limitation}

For almost all deep space missions, the mission profile will impose limits on spacecraft attitude and pointing of spacecraft during certain mission phases. Examples of such mission phases are the launch phase or inner cruise phase where spacecraft attitude is constrained by the trajectory or thermal consideration.there will be coverage holes that can potentially limit the mission planning.
-Difficulty in maintaining link with degraded spacecraft performance: This can include degraded station-keeping capability, degraded star tracker performance, or loss of time reference, will lead to unmanageable acquisition time.

The detection sensitivity is significantly worse at optical frequency, even with the use of high order PPM modulation . The optical receiver sensitivity can further degrade to 20-30 photons/bit under daytime conditions with current receiver technology.

Due to a high distance and difficulty of continuous acquisition and tracking, we can not use laser communication for this mission.


\section{Surface-to-Orbiter} % Rasmus

1. Introduction and considerations for the link
      * (Orbit characteristics and Europa Environment)
1. Link Drivers
   1. Radiation (Europa surf dead zone)
   2. Low Power
   3. Transmission Relay Window
      * Dataload
      * Bitrate
   1. Low Temperature Operation
   2. Line of Sight (viewing angles)
      * Communication while descent maneuver
      * Antenna choice
      * Mechanical stabilizers
1. Link Budget
2. Solution Proposal
3. Drivers to other systems

%   Include in Comm Systems
%   target 8-10 pages
\autsection{Penetrator-Lander Link}{Gustavo Feijóo Carrillo}
%   Through Ice Communications
\subsection{Link Drivers}

\autsubsection{Tethered Link}{Kristian Sloth Lauszus}

One other option for the penetrator to surface communication would be to use an optical fiber connected from the end of the penetrator to the surface station.

\subsubsection{Optical fiber strain}

According to Hook's law the force along the longitudinal axis of the fiber is given by:
\begin{equation}
	 F = k \, \Delta l
\end{equation}
Where $k$ is the elastic constant and $\Delta l$ is the relative deformation or caused perturbation force $F$. This law is fulfilled as long as the deformation does not exceed the elastic limit of the optical fiber which will prevent the fiber from retracting back to its original shape.

By knowing the Young's modulus of the fiber once can calculate the force of the perturbation using the following formula:
\begin{equation}
	F = E_G \, A \frac{\Delta l}{l}
\end{equation}
Where $E_G$ is the Young's modules, $A$ is the area of the optical fiber and $l$ is the length of the optical fiber.

By measuring the force applied to a optical fiber and plotting it versus the strain one can estimate the Young's modulus at the point where the slope of plot is no longer linear. Furthermore this point will also indicate the maximum strain of a given optical fiber.

This strain number is very useful in our case, as this tell us how much the fiber can stretch before it is degraded. A typical number for protected optical fiber is about 3-5 percent\cite{article:optical_fiber_properties,article:optical_fiber_mechanical}. Since the hole above the penetrator will start to freeze up again it is very important that the vertical strain of the of the ice does not exceed this critical value. Fortunately the vertical 

0.3 \% strain

\cite[p. 76]{book:communication}

%An optical tether through the ice is extremely practical. The primary difficulty with deploying an optical tether is the tidal flexing of the ice, which may be as much as 30 m per day. Over the full depth of the ice, this flexure corresponds to about 0.3 \% strain [46]. Using an online calculator provided by Corning Optical Fiber[133], 0.3 \% strain corresponds to a stress of 31 ksi. In testing 386 km of optical fiber, Corning detected no failures at less than 66 ksi [134]. Thus, it is entirely plausible to have an optical tether that connects the surface unit to the bottom of the ice and such a cable would have a mass on the order of a few kilograms, depending on the thickness of the ice [132].

* Optical fiber
	* Pros:
		* Cheap
		* Huge bandwidth
	* Cons:
		* Mechanics % Wheel
			* Complexity, weight, volume
		* Potential break due to ice movement or getting stuck
	* Should be able to resist vertical strain
	* Can be used for redundancy
		* But very risky!

\autsubsection{Ice embedded Repeaters link}{Kristian Sloth Lauszus}

% http://www.unmanned.vt.edu/discovery/reports/VaCAS_2013_01.pdf

* Repeaters along decent path
	* 4 transceivers: 3.5 km pure
	* 14 transceivers: 10 km salty
	* Pros:©
		* Scalable
	* Cons:
		* Complexity
		* Separate RTGs
	* Too complex
		* Better just use one proper designed unit in the penetrator
		* Might be a combinations of repeaters and a big antenna at the lander and probe
		* Will also work better if the probe doesn't move straight

\autsubsection{High Directivity Link}{Ioannis Nisopoulus}
\begin{figure}[htb]
	\centering
	\includegraphics[width=\textwidth]{figures/comms/iceLink-p2p-HighD}
	\caption{ \textit{DRAFT} Sol I.}
	\label{fig:iceLink-p2p-HighD}
\end{figure}

%\autsection{Highly Directional Link}{Nissopoulos Ioannis}

\subsection{Link Description}

\paragraph{Directional antennas}
Directivity is a fundamental antenna parameter, which measures how much power radiated or received in a specific direction from the antenna. A directional antenna is the exact opposite of an omni-directional antenna, which transmits or receives equally from all directions. Directional antennas provide increased performance and reduced interference from unwanted sources, because of their construction and their basic principles. Well known applications for high directional antennas are among others NASA's Deep Space Network (DSN), terrestrial communication links, such as satellite television and cellular repeaters, where high position accuracy and propagation in long distances are needed. In general, high directional antennas are used in every application where high data rates, high gain and reduced interference is mandatory.

The main principle of directional antennas is their ability to concentrate almost all the transmitted power to small beamwidths, instead of illuminating wider solid angles. Saying that, they do not sacrifice any power to unwanted directions, but achieve high gain (more than 10 dBi) for specific solid angles. On the other hand, high directional antennas can reach their targets only through very narrow beamwidths, which can be a significant bottle neck in many applications and especially when tracking in long distances is required. 

Typical types of directional antennas are parabolic antennas, helical antennas, yagi antennas, horn antennas, phased arrays, etc. Directivity and gain of such types of antennas can be affected from many different parameters, regarding their type and application. The number of elements and their spacing, their physical or synthetic aperture, their surrounding environment and the manufacturing material are some of the major criterion that can change the performance of a highly directive antenna considerably. 

\paragraph{Solution description}
In this subsection we will examine if it is possible to use a high directional link for the communication segment between the lander vehicle and the penetrator of our mission. The advantages and disadvantages will be analyzed thoroughly and finally an assessment of this possibility will take place. 

The aforementioned idea is based on a pair of highly directive antennas, which can be used to achieve high rates of data streams and a stable communication channel. One antenna will be installed on lander vehicle's bottom surface looking downwards, while the other one will be placed on the top surface of the penetrator facing upwards, towards the first antenna. The line of sight between these two antennas must always be free of obstacles and we must assure that the beamwidths of the antennas are aligned and cross each other all the time. \textbf{pic} Different types of antennas will be examined in order to find out which one them is the most efficient for the requirements of our mission.

\paragraph{Drivers of Comm Link}
The design of a communication link in an inhospitable environment as Europa's subsurface can be proved especial demanding and difficult, but definitely will not be our only limitation. Space and power consumption is most of the times the number one drivers in all remotely controlled missions. The first priority when an engineer designs a communication link is the selection of the frequency range for the specific mission. In order to do so, engineers have to take into account a lot of factors such as the host environment and the attenuation that implies, the desired bandwidth, the amount of data with the corresponding data rates, etc. 

As we expect, Europa's subsurface consists of several kilometers of ice, which is a factor that limits our choice of frequency range. As described in subsection \ref{sec:ice_losses}, the longer the wavelength we use the better penetration through the ice crust we can achieve. In other words, as we go down in frequency our losses due to ice layers are diminished. Longer wavelengths are also more effective against attenuation due to impurities, because the dimensions of these impurities particles (sulfates, debris rocks, etc.) expected to be in order of millimeters. Nonetheless, the frequency of an antenna is always bonded to its size, by means that lower frequencies correspond to bigger size of antennas. Having in mind our requirements for high directivity, gain and satisfying bandwidth we must keep the size of our antenna reasonable. Other parameters that affect this size can be the type of the antenna. Moreover, due to the limited available space we will provide ways to fold and deploy the antennas wherever it is necessary.  

% \textbf{Obstacles in the near field}

\subsection{Implementation}
In this paragraph, we will examine different type of antennas for the lander vehicle, but also for the penetrating vessel. The main criteria for choosing an antenna type are the size and the ease of folding and deployment, the providing gain and the beamwidth or directivity of the antenna. Other characteristics under examination will be the capability of steering, mechanically or electronically, and the achievable data rates. After the determination of the above, we will perform the link budget analysis for the specified link. 

At first, we will concentrate only on the lander vehicle's antenna and in the next part on penetrator's. In figure \ref{fig:J_spec} can be seen the highest frequency, which is free of Jupiter's radio emissions and it is around 50 MHz, thus this is the lowest frequency choice we can make. The wavelength of this frequency is approximately 6 meters, which is way bigger than 1 meter which is the estimated available space we have under the "belly" of our land vehicle. In order to overcome this problem, we will investigate the case of a slightly higher frequencies, above 150 MHz. 

Firstly, we will present the candidate types of antennas, which will be mounted beneath our lander vehicle. It is difficult to analyze all the different types of directional antennas in the context of this report, thus we will examine three major antenna's "families", which fulfill the requirements discussed above. The first antenna type is Yagi-Uda (or Yagi) antennas together with the log periodic dipole array (LPDA). The former are used mainly from radio amateurs and they are inexpensive but quite efficient antennas for lower frequencies. A distinctive characteristic of Yagi antennas is that their design and construction based most of the times in experimental measurements hands-on experience, rather than well documented formulas and heavy mathematics. Many radio amateurs share their experience and design characteristics in order to improve the basic design as Shintaro Uda and Hidetsugu Yagi had designed in 1962. The later are a special type of antennas, which are quite similar in design with Yagis, although their distinctive characteristic is the very broad bandwidth. Because of this LPDAs are used in UHF terrestrial TV, HF communications, EMC measurements etc.

\paragraph{Yagi-Uda Design}
Yagi-Uda antennas consist of a single driven element connected to the transmitter or receiver and additional parasitic elements. The parasitic elements operate by re-radiating their signals in a slightly different phase in relation to the driven element. In this way the signal is reinforced in some directions and cancelled out in others, so a high directivity is achieved. The length and the spacing of each parasitic element determined by the operational frequency, as well as the total length of the antenna. As it can be seen in figure \ref{Yagi}, a Yagi antenna is linear polarized, parallel to its elements dimension. This is a significant drawback of this design in case the user needs to exploit different polarizations. The solution to this can be the "crossed" Yagi-Uda antenna, which practically is two antennas with two orthogonal polarizations on the same boom. Crossed Yagis provide the capability to receive or transmit in circular polarization with a -3 dB trade off of your signal. If we finally will choose this antenna type for sure we need to consider the crossed version, because the received polarization will have been rotated randomly from the ice crust. LPDAs may look similar to Yagis, although their main principle is quite different. They consist of a number of dipole elements as well, but not all antenna elements are active at any given frequency. In other words, we can imagine a LPDA as an array of dipoles tuned in different frequencies and this is the reason why these antennas hold bandwidth more than few GHz. Another difference is that the main beam of this antenna comes from the "shorter" front. LPDAs used because of their wide bandwidth and not so much because of their gain or their bandwidth. For our purposes, this antenna type will not present so many attractive points, so we will not cover them thoroughly. 

\begin{figure}[htb]
    \centering
    \begin{subfigure}[b]{0.45\textwidth}
        \includegraphics[width=\textwidth]{figures/Yannis/yagi.jpg}
        \caption{A typical Yagi-Uda antenna with 3 directors, 1 driven element and 1 reflectors}\label{Yagi}
    \end{subfigure}
    \begin{subfigure}[b]{0.45\textwidth}
        \includegraphics[width=\textwidth]{figures/Yannis/log2.jpg}
        \caption{The basic components of a log periodic dipole array. The forward direction is to the left in this sketch}\label{log}
    \end{subfigure}
    \caption{}
\end{figure}

As it is mentioned above, the design of a Yagi-Uda antenna is usually approached with an optimization algorithm based on already existed designs. Mentioning that, the free software "Yagi Calculator" (from VK5DJ radio amateur, \url{http://www.vk5dj.com/yagi.html}) was used, which simulates the antenna and has as an output its gain, beamwidth, antenna length, etc. Our major bottleneck using Yagis will be first of all their boom length. According to many designs, the boom length should be at least 2.2 $\lambda$ in order to achieve the best performance out of the antenna. Our available space is approximately 1 meter, thus our antenna boom should be less than one meter, which corresponds practically to frequencies higher than 600 MHz. As it is clear, frequencies higher than 400 MHz present excessive losses and is impossible to be used for our application. Although, we can decrease boom length according to our desires with a trade off in gain and beamwidth. Another issue that we must take into account is the really close distant of ice from the tip of our antenna. This mean that the ice will be inside the near field of our antenna, which can causes serious troubles to our design. The limit between near and far field is given by Fraunhofer's formula $d=\frac{2D^2}{\lambda}$, where D is the largest dimension of the radiator and $\lambda$ is the wavelength of the radio wave. So, we conclude that in order to eliminate or at least reduce the effect of the near field because of the ice, we have to leave a distance equals d before the ice layer. Usually, this distance is very close to the boom length, thus we could claim that we need d=80\% of boom's length, in best case scenario. 

Having the length's limitation in mind the following results came up from the software.
\begin{table}[htb]
\begin{tabular}{| c | c | c | c | c |}
\hline
 f (MHz) & Gain (dBi) & Beamwidth (-3 dB) & Boom Length (mm) & Director No. \\ 
 \hline
 300 & 8.6 & 75 & 515 & 2 \\  
 \hline
 400 & 12 & 65 & 562 & 3\\
 \hline
  430 & 11 & 57 & 700 & 4 \\
 \hline
\end{tabular}
\caption{Potential Yagi designs for different frequencies.}
\label{table: Yagi}
\end{table}

It is shown from table \ref{table: Yagi} that the beamwidths, even at 430 MHz, are fairly wide and it is difficult to define an antenna with this values as absolutely directional. Moreover, in case of a crossed design the power divided into two radiating dipoles, so we have to substract 3 dB from each case.

\paragraph{LPDA Design}
In contrast, LPDAs can be operated over a range of frequencies and over this range its electrical characteristics gain, feed-point impedance, front-to-back ratio, etc. will remain more or less constant. The two most important parameters for LPDA are $\tau$ and $\sigma$, where $\tau$ is the ratio of the length of one element to its next longest neighbor and is constant for a given design for all elements and $\sigma$ is known as the “relative spacing constant” and along with $\tau$ determines the angle of the antenna’s apex, a. 

The design process starts choosing the desired gain in dBi and from \ref{log_gain} $\tau$ and $\sigma$ are determined. A good choice of gain for this type of antennas is around 7 dBi, which is also quite common in commercial applications. Having in mind that $\tau$ and $\sigma$ define the number of elements, thus the size of antenna's boom, we want to keep them at reasonable low numbers. Observing figure \ref{log_gain}, we choose $\tau=0.9$ and $\sigma0.07$ in order to have 7 dBi gain. Also, our operational bandwidth will be from 200 to 400 MHz.

In order to find out the total length of the antenna and its number of elements the apex angle must be calculated. The bandwidth for this design is computed afterwards and from this the number of elements N and the total length of the structure are computed. Finally, the average characteristic impedance of the elements $Z_{a}$ is calculated. In the next lines these calculations take place, according to \cite{balanis}, while the MATLAB code can be found in appendix \ref{matlab}.

\begin{figure}[htb]
\centering
\includegraphics[width=1\textwidth]{figures/Yannis/Log_gain}
\caption{The parameters tau and sigma can be chosen from this graph. The line for optimum sigma is for those designers who want maximum gain\cite{balanis,Log}}
\label{log_gain}
\end{figure}

\begin{subequations}
\begin{align}
    a&=tan^{-1}[\frac{1-\tau}{4 \sigma}]=tan^{-1}[\frac{1-0.9}{4 \cdot 0.07}]=19.65^{\circ} \\
    B_{s}&=B \cdot B_{ar}=\frac{f_{high}}{f_{low}} (1.1+7.7(1-\tau)^2 \cdot cot(a))= \frac{400}{200} \cdot 1.316=2.631 \\
    N&=1+\frac{ln(B_{s})}{ln(\frac{1}{\tau})}=10.2 \approx 10 \ elements \\
    L&=\frac{c}{f_{low}} (1-\frac{1}{B_{s}}) \frac{cot(a)}{4}=0.6509 \ meters \\
    Z_{a}&=120(ln(\frac{l_{max}}{d})-2.25)=120 \cdot (ln(\frac{0.75}{0.019})-2.25)=171 \ \Omega
\end{align}
\label{eq: LPDA}
\end{subequations}

,where $l_{max}=\frac{c}{2 \cdot f_{min}}$ is the length of the maximum element and d=1.9 cm, which is the outer diameter of each element. Usually, a balun is needed in order to regulate the input impedance to desired values. 

LPDAs have a wide 3 dB beamwidth, which most of the times calculated through simulations, measurements or empirical tables. One f these tables is shown in figure \ref{hpbw_log}. Because of this wide beamwidth, LPDAs are not the best choice for our directional application, although if directivity is not our main concern there are manufacturers that sell very appealing, foldable antennas, some of them are presented in appendix \ref{LPDA1}. Finally, LPDAs are linear polarized, thus we have to mount two different, orthogonal antenna booms or one crossed design to receive properly the random polarization's orientation. Of course, this mean an extra 3 dB loss. A summary of the most suitable parameters of an LPDA for our mission can be found in table \ref{table: LPDA}.

\begin{figure}[htb]
\centering
    \captionsetup{width=0.8\textwidth}
        \centering
        \includegraphics[width=1\textwidth]{figures/Yannis/HPBW_Log.jpg}
        \caption{Half power beamwidth for commercial log-periodic dipole arrays\cite{balanis}}
        \label{hpbw_log}
\end{figure}
\begin{figure}[htb]
\centering
        \captionsetup{width=0.8\textwidth}
        \centering
        \includegraphics[width=1\linewidth]{figures/Yannis/VSWR_log.jpg}
        \caption{VSWR (Voltage Standing Wave Ratio) for commercial log-periodic dipole arrays\cite{balanis}}
        \label{vswr_lpda}
\end{figure}

\begin{table}[htb]
\begin{tabular}{| c | c | c | c | c | c | c | c |}
\hline
 $\tau$ & $\sigma$ & $f_{high}$ (MHz) & $f_{low}$ (MHz) & Gain (dBi) & HPBW & L (mm) & N \\ 
 \hline
 0.9 & 0.07 & 400 & 200 & 7 & (E)60$^\circ$-(H)100$^\circ$ & 650 & 10 \\
 \hline
\end{tabular}
\caption{Potential Log-Periodic Dipole Array design.}
\label{table: LPDA}
\end{table}

\paragraph{Helical Antennas Design}
Another type of antennas with a lot of potential in our mission is helical antenna. This antenna type is consisting of a conducting wire wound in the form of a helix, which in most cases is mounted over a ground plane. There are two modes that a helical antenna can operate and these are the normal mode and the axial. At normal mode the antenna resembles a monopole or dipole, so it has omnidirectional characteristics and linear polarization. We are interested in directional antennas, thus we will focus on the axial mode. The axial mode provides a directional antenna beam, radiating off the ends of the helix along the antenna's axis (end-fire). Moreover, the latter mode creates a circular polarized wave, which is mandatory for our application. A typical helical antenna with its radiation pattern is depicted in figure \ref{helical}.

\begin{figure}[htb]
    \centering
    \begin{subfigure}[b]{0.45\textwidth}
        \includegraphics[width=\textwidth]{figures/Yannis/helix.jpg}
        \caption{A typical representation of a helical antenna. The most important parameters, which determine its radiation pattern are shown\cite{helix}}\label{helix1}
    \end{subfigure}
    \begin{subfigure}[b]{0.45\textwidth}
        \includegraphics[width=\textwidth]{figures/Yannis/H2.png}
        \caption{Radiation pattern of helical antenna when operating in axial mode\cite{helix}}\label{helix_pat}
    \end{subfigure}
    \caption{Helical antenna sketch and radiation pattern.}\label{helical}
\end{figure}

The main constraints in building and using a helical antenna for our purposes would be again its total length, the beamwidth, the gain and the remaining distance to the ice layer. In order to examine the behaviour of such antennas the definition of the basic characteristics is needed. As it is shown in figure \ref{helix1} D represents antennas diameter, N turns of wire, d radius of wire and S the gap between turns. The overall length is $L=N \cdot S$ and the circumference of a single turn is $C=\pi \cdot D$. Last of the mechanical parameters are pitch angle of helix $a$ and R the reflector size, which must be at least 3/4 of the operating wavelength $\lambda$. In addition, we must take into account some experimentally derived guidelines, such as the value of pitch angle $a$ should be between $12^\circ$ and $14^\circ$. If $a$ is not in the specified range anomalies in the performance can occur. Also, the following formulas (eq. \ref{helical_eq}) are valid for $N \geq 3$ and $<3/4C/\lambda<4/3$ \cite{balanis2}.

For our design a trade off between total length and the parameters of frequency, C and S must be made. Generally, more turns (higher N) provide higher directivity $D_{0}$ and also the frequency dependence of the input impedance is smaller, but on the other hand the length can be disturbing big. For our mission we experimented with a MATLAB code (appendix \ref{matlab}) in different specifications and we ended up with the most suitable, as they presented in tables \ref{table: Helicals1} and \ref{table: Helicals2}. Additionally, we examined the possibility of using an array of helical antennas, but it can be seen for our frequencies it is more efficient (mainly gain-wise) to use in element. The following equations were used for this design \cite{balanis2}.

\begin{subequations}
\begin{align}
    &C=\pi \cdot D \\
    &a=tan^{-1}(\frac{S}{C}) \\
    &L=N \cdot S \\
    &R=\frac{3}{4} \lambda \\
    &Z_{in}=140 \cdot (\frac{C}{\lambda}) \\
    &HPBW=\frac{52 \cdot \lambda^{\frac{3}{2}}}{C \cdot \sqrt{N \cdot S}} \\
    &D_{0}=\frac{52 \cdot \lambda^{\frac{3}{2}}}{C \cdot \sqrt{N \cdot S}} \\
    &AR=\frac{2N+1}{2N} 
\end{align}
\label{helical_eq}
\end{subequations}

Except for a single helical antenna, we are simulated the behaviour of an array of helical antennas to examine if it is beneficial to use this implementation. The extra equations for an array are shown below (eq. \ref{helical_array}). Two approaches for the directivity were used, a simple one ($D_{array}$) and one taking into account the Hansen-Woodyard ($D_{HW}$) approximation. As it is presented in \ref{table: Helicals1} and \ref{table: Helicals2}, the array design it is not that useful for our frequencies, because in order to have better performance than the single helical antenna, more than 5 elements are needed.

\begin{subequations}
\begin{align}
    &dis=1.5 \cdot \lambda \\
    &D_{array}=N_{elem} \cdot (1+\frac{L}{dis})(\frac{dis}{\lambda}) \\
    &D_{HW}=1.805(N_{elem}(1+\frac{L}{dis})(\frac{dis}{\lambda})) 
\end{align}
\label{helical_array}
\end{subequations}
, where $N_{elem}$ is the number of array's elements and $dis$ is the distance between them. 

\begin{table}[htb]
\centering
\begin{tabular}{| c | c | c | c | c |}
\hline
 f (MHz) & Gain (dBi) & HPBW & Length (mm) & $Gain_{array}$ \\ 
 \hline
 300 & 7.8 & 64.5$^\circ$ & 800 & 16.6 \\  
 \hline
 340 & 23.6 & 37$^\circ$ & 1000 & 19\\
 \hline
  400 & 24.71 & 36.23$^\circ$ & 876 & 19.2 \\
 \hline
\end{tabular}
\caption{Helical antenna with 4 turns (N=4) specifications. The last column shows the values of gain in dBi for an array of 4 elements.}
\label{table: Helicals1}
\end{table}

\begin{table}[htb]
\centering
\begin{tabular}{| c | c | c | c | c |}
\hline
 f (MHz) & Gain (dBi) & HPBW & Length (mm) & $Gain_{array}$ \\ 
 \hline
 300 & 5.8 & 74.5$^\circ$ & 600 & 15.1 \\  
 \hline
 340 & 17.7 & 42.8$^\circ$ & 750 & 16.9\\
 \hline
  400 & 18.5 & 41.8$^\circ$ & 657 & 17.2 \\
 \hline
\end{tabular}
\caption{Helical antenna with 3 turns (N=3) specifications. The last column shows the values of gain in dBi for an array of 4 elements.}
\label{table: Helicals2}
\end{table}
A more detailed table for helical antennas can be found in appendix \ref{table: Helicals}.

As it is shown from tables \ref{table: Helicals1} and \ref{table: Helicals2}, the most suitable case is in frequency 340 MHz, where the total length of the antenna is 750 mm, the maximum gain is 17.7 dBi and the half power beamwidth is 42.8$^\circ$. There is also 250 mm free space between the end of the antenna and the ice, which could be enough to ignore ice as a near field obstacle. The gain exceeds our expectation, but bibliography states that these values may be exaggerated. But even in these case, the gain is still more than enough and helical antennas can receive or transmit circular polarization without the need of any modifications or additional losses. So, up to now it seems that a helical antenna is the most appealing option.

\paragraph{Parabolic Dish Design}
Parabolic dish antennas are very directional, high gain antennas, which are used mainly for the frequency range 2-15 GHz. For lower frequencies, like ours, the dish diameter increases to several meters. The basic formulas, which describe a dish antenna behaviour are presented below (eq. \ref{dish_eq}).

\begin{subequations}
\begin{align}
    &G=10 \cdot log_{10}(k \cdot (\pi \frac{D}{\lambda})^2) \\
    &HPBW=60 \cdot \frac{\lambda}{D}
\end{align}
\label{dish_eq}
\end{subequations}

\begin{table}[htb]
\centering
\begin{tabular}{| c | c | c | c | c |}
\hline
 f (MHz) & Gain (dBi) & HPBW & Diameter (mm) & k \\ 
 \hline
 300 & 7.2 & 66.6$^\circ$ & 900 & 0.66 \\  
 \hline
 340 & 8.3 & 58.8$^\circ$ & 900 & 0.66\\
 \hline
  400 & 9.7 & 50$^\circ$ & 900 & 0.66\\
 \hline
\end{tabular}
\caption{Parabolic antenna's specification for different frequencies.}
\label{table: dish}
\end{table}

, where G is the gain in dBi, D is the diameter of the parabolic reflector in metres and k is the efficiency factor which is generally around 50\% to 70\%. The parabolic reflector antenna gain efficiency is dependent upon a variety of factors, which are all multiplied together to give the overall efficiency. These factors are radiation efficiency, aperture taper efficiency, spillover efficiency, surface error, cross polarization, etc. The design of a parabolic antenna is way more complex, so in order to achieve very good results optimization of the feed point and a choice between different type of dish implementations must take place. By different type of dish implementations is meant, the Cassegrain, Gregorian or off-axis design. In the context of this section, we will not go any further because as it is clear from the results shown in table \ref{table: dish}, a parabolic antenna in this frequency range and under the restriction in diameter can not be really directional. Although, the gain levels are sufficient if we would like to go with a more wide HPBW.  

\subsection{Low Directivity Link}
\begin{figure}[htb]
	\centering
	\includegraphics[width=\textwidth]{figures/comms/iceLink-p2p-LowD}
	\caption{ \textit{DRAFT} Sol II.}
	\label{fig:iceLink-p2p-LowD}
\end{figure}

\subsection{Relay Link}
\begin{figure}[htb]
	\centering
	\includegraphics[width=\textwidth]{figures/comms/iceLink-relay}
	\caption{ \textit{DRAFT} Sol III relays.}
	\label{fig:iceLink-relay}
\end{figure}
	

%\section{Probe-to-Penetrator}

%\subsection{High Directional Link}

\autsection{Highly Directional Link}{Nissopoulos Ioannis}

\subsection{Link Description}

\paragraph{Directional antennas}
Directivity is a fundamental antenna parameter, which measures how much power radiated or received in a specific direction from the antenna. A directional antenna is the exact opposite of an omni-directional antenna, which transmits or receives equally from all directions. Directional antennas provide increased performance and reduced interference from unwanted sources, because of their construction and their basic principles. Well known applications for high directional antennas are among others NASA's Deep Space Network (DSN), terrestrial communication links, such as satellite television and cellular repeaters, where high position accuracy and propagation in long distances are needed. In general, high directional antennas are used in every application where high data rates, high gain and reduced interference is mandatory.

The main principle of directional antennas is their ability to concentrate almost all the transmitted power to small beamwidths, instead of illuminating wider solid angles. Saying that, they do not sacrifice any power to unwanted directions, but achieve high gain (more than 10 dBi) for specific solid angles. On the other hand, high directional antennas can reach their targets only through very narrow beamwidths, which can be a significant bottle neck in many applications and especially when tracking in long distances is required. 

Typical types of directional antennas are parabolic antennas, helical antennas, yagi antennas, horn antennas, phased arrays, etc. Directivity and gain of such types of antennas can be affected from many different parameters, regarding their type and application. The number of elements and their spacing, their physical or synthetic aperture, their surrounding environment and the manufacturing material are some of the major criterion that can change the performance of a highly directive antenna considerably. 

\paragraph{Solution description}
In this subsection we will examine if it is possible to use a high directional link for the communication segment between the lander vehicle and the penetrator of our mission. The advantages and disadvantages will be analyzed thoroughly and finally an assessment of this possibility will take place. 

The aforementioned idea is based on a pair of highly directive antennas, which can be used to achieve high rates of data streams and a stable communication channel. One antenna will be installed on lander vehicle's bottom surface looking downwards, while the other one will be placed on the top surface of the penetrator facing upwards, towards the first antenna. The line of sight between these two antennas must always be free of obstacles and we must assure that the beamwidths of the antennas are aligned and cross each other all the time. \textbf{pic} Different types of antennas will be examined in order to find out which one them is the most efficient for the requirements of our mission.

\paragraph{Drivers of Comm Link}
The design of a communication link in an inhospitable environment as Europa's subsurface can be proved especial demanding and difficult, but definitely will not be our only limitation. Space and power consumption is most of the times the number one drivers in all remotely controlled missions. The first priority when an engineer designs a communication link is the selection of the frequency range for the specific mission. In order to do so, engineers have to take into account a lot of factors such as the host environment and the attenuation that implies, the desired bandwidth, the amount of data with the corresponding data rates, etc. 

As we expect, Europa's subsurface consists of several kilometers of ice, which is a factor that limits our choice of frequency range. As described in subsection \ref{sec:ice_losses}, the longer the wavelength we use the better penetration through the ice crust we can achieve. In other words, as we go down in frequency our losses due to ice layers are diminished. Longer wavelengths are also more effective against attenuation due to impurities, because the dimensions of these impurities particles (sulfates, debris rocks, etc.) expected to be in order of millimeters. Nonetheless, the frequency of an antenna is always bonded to its size, by means that lower frequencies correspond to bigger size of antennas. Having in mind our requirements for high directivity, gain and satisfying bandwidth we must keep the size of our antenna reasonable. Other parameters that affect this size can be the type of the antenna. Moreover, due to the limited available space we will provide ways to fold and deploy the antennas wherever it is necessary.  

% \textbf{Obstacles in the near field}

\subsection{Implementation}
In this paragraph, we will examine different type of antennas for the lander vehicle, but also for the penetrating vessel. The main criteria for choosing an antenna type are the size and the ease of folding and deployment, the providing gain and the beamwidth or directivity of the antenna. Other characteristics under examination will be the capability of steering, mechanically or electronically, and the achievable data rates. After the determination of the above, we will perform the link budget analysis for the specified link. 

At first, we will concentrate only on the lander vehicle's antenna and in the next part on penetrator's. In figure \ref{fig:J_spec} can be seen the highest frequency, which is free of Jupiter's radio emissions and it is around 50 MHz, thus this is the lowest frequency choice we can make. The wavelength of this frequency is approximately 6 meters, which is way bigger than 1 meter which is the estimated available space we have under the "belly" of our land vehicle. In order to overcome this problem, we will investigate the case of a slightly higher frequencies, above 150 MHz. 

Firstly, we will present the candidate types of antennas, which will be mounted beneath our lander vehicle. It is difficult to analyze all the different types of directional antennas in the context of this report, thus we will examine three major antenna's "families", which fulfill the requirements discussed above. The first antenna type is Yagi-Uda (or Yagi) antennas together with the log periodic dipole array (LPDA). The former are used mainly from radio amateurs and they are inexpensive but quite efficient antennas for lower frequencies. A distinctive characteristic of Yagi antennas is that their design and construction based most of the times in experimental measurements hands-on experience, rather than well documented formulas and heavy mathematics. Many radio amateurs share their experience and design characteristics in order to improve the basic design as Shintaro Uda and Hidetsugu Yagi had designed in 1962. The later are a special type of antennas, which are quite similar in design with Yagis, although their distinctive characteristic is the very broad bandwidth. Because of this LPDAs are used in UHF terrestrial TV, HF communications, EMC measurements etc.

\paragraph{Yagi-Uda Design}
Yagi-Uda antennas consist of a single driven element connected to the transmitter or receiver and additional parasitic elements. The parasitic elements operate by re-radiating their signals in a slightly different phase in relation to the driven element. In this way the signal is reinforced in some directions and cancelled out in others, so a high directivity is achieved. The length and the spacing of each parasitic element determined by the operational frequency, as well as the total length of the antenna. As it can be seen in figure \ref{Yagi}, a Yagi antenna is linear polarized, parallel to its elements dimension. This is a significant drawback of this design in case the user needs to exploit different polarizations. The solution to this can be the "crossed" Yagi-Uda antenna, which practically is two antennas with two orthogonal polarizations on the same boom. Crossed Yagis provide the capability to receive or transmit in circular polarization with a -3 dB trade off of your signal. If we finally will choose this antenna type for sure we need to consider the crossed version, because the received polarization will have been rotated randomly from the ice crust. LPDAs may look similar to Yagis, although their main principle is quite different. They consist of a number of dipole elements as well, but not all antenna elements are active at any given frequency. In other words, we can imagine a LPDA as an array of dipoles tuned in different frequencies and this is the reason why these antennas hold bandwidth more than few GHz. Another difference is that the main beam of this antenna comes from the "shorter" front. LPDAs used because of their wide bandwidth and not so much because of their gain or their bandwidth. For our purposes, this antenna type will not present so many attractive points, so we will not cover them thoroughly. 

\begin{figure}[htb]
    \centering
    \begin{subfigure}[b]{0.45\textwidth}
        \includegraphics[width=\textwidth]{figures/Yannis/yagi.jpg}
        \caption{A typical Yagi-Uda antenna with 3 directors, 1 driven element and 1 reflectors}\label{Yagi}
    \end{subfigure}
    \begin{subfigure}[b]{0.45\textwidth}
        \includegraphics[width=\textwidth]{figures/Yannis/log2.jpg}
        \caption{The basic components of a log periodic dipole array. The forward direction is to the left in this sketch}\label{log}
    \end{subfigure}
    \caption{}
\end{figure}

As it is mentioned above, the design of a Yagi-Uda antenna is usually approached with an optimization algorithm based on already existed designs. Mentioning that, the free software "Yagi Calculator" (from VK5DJ radio amateur, \url{http://www.vk5dj.com/yagi.html}) was used, which simulates the antenna and has as an output its gain, beamwidth, antenna length, etc. Our major bottleneck using Yagis will be first of all their boom length. According to many designs, the boom length should be at least 2.2 $\lambda$ in order to achieve the best performance out of the antenna. Our available space is approximately 1 meter, thus our antenna boom should be less than one meter, which corresponds practically to frequencies higher than 600 MHz. As it is clear, frequencies higher than 400 MHz present excessive losses and is impossible to be used for our application. Although, we can decrease boom length according to our desires with a trade off in gain and beamwidth. Another issue that we must take into account is the really close distant of ice from the tip of our antenna. This mean that the ice will be inside the near field of our antenna, which can causes serious troubles to our design. The limit between near and far field is given by Fraunhofer's formula $d=\frac{2D^2}{\lambda}$, where D is the largest dimension of the radiator and $\lambda$ is the wavelength of the radio wave. So, we conclude that in order to eliminate or at least reduce the effect of the near field because of the ice, we have to leave a distance equals d before the ice layer. Usually, this distance is very close to the boom length, thus we could claim that we need d=80\% of boom's length, in best case scenario. 

Having the length's limitation in mind the following results came up from the software.
\begin{table}[htb]
\begin{tabular}{| c | c | c | c | c |}
\hline
 f (MHz) & Gain (dBi) & Beamwidth (-3 dB) & Boom Length (mm) & Director No. \\ 
 \hline
 300 & 8.6 & 75 & 515 & 2 \\  
 \hline
 400 & 12 & 65 & 562 & 3\\
 \hline
  430 & 11 & 57 & 700 & 4 \\
 \hline
\end{tabular}
\caption{Potential Yagi designs for different frequencies.}
\label{table: Yagi}
\end{table}

It is shown from table \ref{table: Yagi} that the beamwidths, even at 430 MHz, are fairly wide and it is difficult to define an antenna with this values as absolutely directional. Moreover, in case of a crossed design the power divided into two radiating dipoles, so we have to substract 3 dB from each case.

\paragraph{LPDA Design}
In contrast, LPDAs can be operated over a range of frequencies and over this range its electrical characteristics gain, feed-point impedance, front-to-back ratio, etc. will remain more or less constant. The two most important parameters for LPDA are $\tau$ and $\sigma$, where $\tau$ is the ratio of the length of one element to its next longest neighbor and is constant for a given design for all elements and $\sigma$ is known as the “relative spacing constant” and along with $\tau$ determines the angle of the antenna’s apex, a. 

The design process starts choosing the desired gain in dBi and from \ref{log_gain} $\tau$ and $\sigma$ are determined. A good choice of gain for this type of antennas is around 7 dBi, which is also quite common in commercial applications. Having in mind that $\tau$ and $\sigma$ define the number of elements, thus the size of antenna's boom, we want to keep them at reasonable low numbers. Observing figure \ref{log_gain}, we choose $\tau=0.9$ and $\sigma0.07$ in order to have 7 dBi gain. Also, our operational bandwidth will be from 200 to 400 MHz.

In order to find out the total length of the antenna and its number of elements the apex angle must be calculated. The bandwidth for this design is computed afterwards and from this the number of elements N and the total length of the structure are computed. Finally, the average characteristic impedance of the elements $Z_{a}$ is calculated. In the next lines these calculations take place, according to \cite{balanis}, while the MATLAB code can be found in appendix \ref{matlab}.

\begin{figure}[htb]
\centering
\includegraphics[width=1\textwidth]{figures/Yannis/Log_gain}
\caption{The parameters tau and sigma can be chosen from this graph. The line for optimum sigma is for those designers who want maximum gain\cite{balanis,Log}}
\label{log_gain}
\end{figure}

\begin{subequations}
\begin{align}
    a&=tan^{-1}[\frac{1-\tau}{4 \sigma}]=tan^{-1}[\frac{1-0.9}{4 \cdot 0.07}]=19.65^{\circ} \\
    B_{s}&=B \cdot B_{ar}=\frac{f_{high}}{f_{low}} (1.1+7.7(1-\tau)^2 \cdot cot(a))= \frac{400}{200} \cdot 1.316=2.631 \\
    N&=1+\frac{ln(B_{s})}{ln(\frac{1}{\tau})}=10.2 \approx 10 \ elements \\
    L&=\frac{c}{f_{low}} (1-\frac{1}{B_{s}}) \frac{cot(a)}{4}=0.6509 \ meters \\
    Z_{a}&=120(ln(\frac{l_{max}}{d})-2.25)=120 \cdot (ln(\frac{0.75}{0.019})-2.25)=171 \ \Omega
\end{align}
\label{eq: LPDA}
\end{subequations}

,where $l_{max}=\frac{c}{2 \cdot f_{min}}$ is the length of the maximum element and d=1.9 cm, which is the outer diameter of each element. Usually, a balun is needed in order to regulate the input impedance to desired values. 

LPDAs have a wide 3 dB beamwidth, which most of the times calculated through simulations, measurements or empirical tables. One f these tables is shown in figure \ref{hpbw_log}. Because of this wide beamwidth, LPDAs are not the best choice for our directional application, although if directivity is not our main concern there are manufacturers that sell very appealing, foldable antennas, some of them are presented in appendix \ref{LPDA1}. Finally, LPDAs are linear polarized, thus we have to mount two different, orthogonal antenna booms or one crossed design to receive properly the random polarization's orientation. Of course, this mean an extra 3 dB loss. A summary of the most suitable parameters of an LPDA for our mission can be found in table \ref{table: LPDA}.

\begin{figure}[htb]
\centering
    \captionsetup{width=0.8\textwidth}
        \centering
        \includegraphics[width=1\textwidth]{figures/Yannis/HPBW_Log.jpg}
        \caption{Half power beamwidth for commercial log-periodic dipole arrays\cite{balanis}}
        \label{hpbw_log}
\end{figure}
\begin{figure}[htb]
\centering
        \captionsetup{width=0.8\textwidth}
        \centering
        \includegraphics[width=1\linewidth]{figures/Yannis/VSWR_log.jpg}
        \caption{VSWR (Voltage Standing Wave Ratio) for commercial log-periodic dipole arrays\cite{balanis}}
        \label{vswr_lpda}
\end{figure}

\begin{table}[htb]
\begin{tabular}{| c | c | c | c | c | c | c | c |}
\hline
 $\tau$ & $\sigma$ & $f_{high}$ (MHz) & $f_{low}$ (MHz) & Gain (dBi) & HPBW & L (mm) & N \\ 
 \hline
 0.9 & 0.07 & 400 & 200 & 7 & (E)60$^\circ$-(H)100$^\circ$ & 650 & 10 \\
 \hline
\end{tabular}
\caption{Potential Log-Periodic Dipole Array design.}
\label{table: LPDA}
\end{table}

\paragraph{Helical Antennas Design}
Another type of antennas with a lot of potential in our mission is helical antenna. This antenna type is consisting of a conducting wire wound in the form of a helix, which in most cases is mounted over a ground plane. There are two modes that a helical antenna can operate and these are the normal mode and the axial. At normal mode the antenna resembles a monopole or dipole, so it has omnidirectional characteristics and linear polarization. We are interested in directional antennas, thus we will focus on the axial mode. The axial mode provides a directional antenna beam, radiating off the ends of the helix along the antenna's axis (end-fire). Moreover, the latter mode creates a circular polarized wave, which is mandatory for our application. A typical helical antenna with its radiation pattern is depicted in figure \ref{helical}.

\begin{figure}[htb]
    \centering
    \begin{subfigure}[b]{0.45\textwidth}
        \includegraphics[width=\textwidth]{figures/Yannis/helix.jpg}
        \caption{A typical representation of a helical antenna. The most important parameters, which determine its radiation pattern are shown\cite{helix}}\label{helix1}
    \end{subfigure}
    \begin{subfigure}[b]{0.45\textwidth}
        \includegraphics[width=\textwidth]{figures/Yannis/H2.png}
        \caption{Radiation pattern of helical antenna when operating in axial mode\cite{helix}}\label{helix_pat}
    \end{subfigure}
    \caption{Helical antenna sketch and radiation pattern.}\label{helical}
\end{figure}

The main constraints in building and using a helical antenna for our purposes would be again its total length, the beamwidth, the gain and the remaining distance to the ice layer. In order to examine the behaviour of such antennas the definition of the basic characteristics is needed. As it is shown in figure \ref{helix1} D represents antennas diameter, N turns of wire, d radius of wire and S the gap between turns. The overall length is $L=N \cdot S$ and the circumference of a single turn is $C=\pi \cdot D$. Last of the mechanical parameters are pitch angle of helix $a$ and R the reflector size, which must be at least 3/4 of the operating wavelength $\lambda$. In addition, we must take into account some experimentally derived guidelines, such as the value of pitch angle $a$ should be between $12^\circ$ and $14^\circ$. If $a$ is not in the specified range anomalies in the performance can occur. Also, the following formulas (eq. \ref{helical_eq}) are valid for $N \geq 3$ and $<3/4C/\lambda<4/3$ \cite{balanis2}.

For our design a trade off between total length and the parameters of frequency, C and S must be made. Generally, more turns (higher N) provide higher directivity $D_{0}$ and also the frequency dependence of the input impedance is smaller, but on the other hand the length can be disturbing big. For our mission we experimented with a MATLAB code (appendix \ref{matlab}) in different specifications and we ended up with the most suitable, as they presented in tables \ref{table: Helicals1} and \ref{table: Helicals2}. Additionally, we examined the possibility of using an array of helical antennas, but it can be seen for our frequencies it is more efficient (mainly gain-wise) to use in element. The following equations were used for this design \cite{balanis2}.

\begin{subequations}
\begin{align}
    &C=\pi \cdot D \\
    &a=tan^{-1}(\frac{S}{C}) \\
    &L=N \cdot S \\
    &R=\frac{3}{4} \lambda \\
    &Z_{in}=140 \cdot (\frac{C}{\lambda}) \\
    &HPBW=\frac{52 \cdot \lambda^{\frac{3}{2}}}{C \cdot \sqrt{N \cdot S}} \\
    &D_{0}=\frac{52 \cdot \lambda^{\frac{3}{2}}}{C \cdot \sqrt{N \cdot S}} \\
    &AR=\frac{2N+1}{2N} 
\end{align}
\label{helical_eq}
\end{subequations}

Except for a single helical antenna, we are simulated the behaviour of an array of helical antennas to examine if it is beneficial to use this implementation. The extra equations for an array are shown below (eq. \ref{helical_array}). Two approaches for the directivity were used, a simple one ($D_{array}$) and one taking into account the Hansen-Woodyard ($D_{HW}$) approximation. As it is presented in \ref{table: Helicals1} and \ref{table: Helicals2}, the array design it is not that useful for our frequencies, because in order to have better performance than the single helical antenna, more than 5 elements are needed.

\begin{subequations}
\begin{align}
    &dis=1.5 \cdot \lambda \\
    &D_{array}=N_{elem} \cdot (1+\frac{L}{dis})(\frac{dis}{\lambda}) \\
    &D_{HW}=1.805(N_{elem}(1+\frac{L}{dis})(\frac{dis}{\lambda})) 
\end{align}
\label{helical_array}
\end{subequations}
, where $N_{elem}$ is the number of array's elements and $dis$ is the distance between them. 

\begin{table}[htb]
\centering
\begin{tabular}{| c | c | c | c | c |}
\hline
 f (MHz) & Gain (dBi) & HPBW & Length (mm) & $Gain_{array}$ \\ 
 \hline
 300 & 7.8 & 64.5$^\circ$ & 800 & 16.6 \\  
 \hline
 340 & 23.6 & 37$^\circ$ & 1000 & 19\\
 \hline
  400 & 24.71 & 36.23$^\circ$ & 876 & 19.2 \\
 \hline
\end{tabular}
\caption{Helical antenna with 4 turns (N=4) specifications. The last column shows the values of gain in dBi for an array of 4 elements.}
\label{table: Helicals1}
\end{table}

\begin{table}[htb]
\centering
\begin{tabular}{| c | c | c | c | c |}
\hline
 f (MHz) & Gain (dBi) & HPBW & Length (mm) & $Gain_{array}$ \\ 
 \hline
 300 & 5.8 & 74.5$^\circ$ & 600 & 15.1 \\  
 \hline
 340 & 17.7 & 42.8$^\circ$ & 750 & 16.9\\
 \hline
  400 & 18.5 & 41.8$^\circ$ & 657 & 17.2 \\
 \hline
\end{tabular}
\caption{Helical antenna with 3 turns (N=3) specifications. The last column shows the values of gain in dBi for an array of 4 elements.}
\label{table: Helicals2}
\end{table}
A more detailed table for helical antennas can be found in appendix \ref{table: Helicals}.

As it is shown from tables \ref{table: Helicals1} and \ref{table: Helicals2}, the most suitable case is in frequency 340 MHz, where the total length of the antenna is 750 mm, the maximum gain is 17.7 dBi and the half power beamwidth is 42.8$^\circ$. There is also 250 mm free space between the end of the antenna and the ice, which could be enough to ignore ice as a near field obstacle. The gain exceeds our expectation, but bibliography states that these values may be exaggerated. But even in these case, the gain is still more than enough and helical antennas can receive or transmit circular polarization without the need of any modifications or additional losses. So, up to now it seems that a helical antenna is the most appealing option.

\paragraph{Parabolic Dish Design}
Parabolic dish antennas are very directional, high gain antennas, which are used mainly for the frequency range 2-15 GHz. For lower frequencies, like ours, the dish diameter increases to several meters. The basic formulas, which describe a dish antenna behaviour are presented below (eq. \ref{dish_eq}).

\begin{subequations}
\begin{align}
    &G=10 \cdot log_{10}(k \cdot (\pi \frac{D}{\lambda})^2) \\
    &HPBW=60 \cdot \frac{\lambda}{D}
\end{align}
\label{dish_eq}
\end{subequations}

\begin{table}[htb]
\centering
\begin{tabular}{| c | c | c | c | c |}
\hline
 f (MHz) & Gain (dBi) & HPBW & Diameter (mm) & k \\ 
 \hline
 300 & 7.2 & 66.6$^\circ$ & 900 & 0.66 \\  
 \hline
 340 & 8.3 & 58.8$^\circ$ & 900 & 0.66\\
 \hline
  400 & 9.7 & 50$^\circ$ & 900 & 0.66\\
 \hline
\end{tabular}
\caption{Parabolic antenna's specification for different frequencies.}
\label{table: dish}
\end{table}

, where G is the gain in dBi, D is the diameter of the parabolic reflector in metres and k is the efficiency factor which is generally around 50\% to 70\%. The parabolic reflector antenna gain efficiency is dependent upon a variety of factors, which are all multiplied together to give the overall efficiency. These factors are radiation efficiency, aperture taper efficiency, spillover efficiency, surface error, cross polarization, etc. The design of a parabolic antenna is way more complex, so in order to achieve very good results optimization of the feed point and a choice between different type of dish implementations must take place. By different type of dish implementations is meant, the Cassegrain, Gregorian or off-axis design. In the context of this section, we will not go any further because as it is clear from the results shown in table \ref{table: dish}, a parabolic antenna in this frequency range and under the restriction in diameter can not be really directional. Although, the gain levels are sufficient if we would like to go with a more wide HPBW.   %Isn't this supposed to be a subsection of the penetrator to lander link?

\autsection{Communication Systems Proposal}{Kristian Sloth Lauszus, Rasmus Lundby Pedersen, and Gustavo Feijóo Carrillo}

==== Trans Ice Communications Presentation
    -Fiber optical cables are heavy.
    -Repeaters require RTGs with a high interval. Number of repeaters is highly depending on ice conditinos
    -One-to-one communication link through the ice is a better alternative.
CONCLUSION: Fiber optical cable is not a good option, cable is going to be enormous.
            One-to-one communication must be researched further - more details are required.
            Repeater communication must be researched further.
            
==== Communications Presentation
-Ice link:
-Surface relay:
-Ground control link:
One-to-one communication, assuming poor ice conditions, link is possible
Surface to orbiter. Small tracking possibilities, low gain antenna (LGA) is used.
3m high gain antenna.
CONCLUSION: One-to-one communication must be researched further. More details are needed
            The antenna transmitting down through the ice must be aligned. High risk, as landing site will probably not be flat.

* Repeaters along decent path
	* Increases a lot when introducing salinity
	* Pros:
		* Scalable
	* Cons:
		* Complexity
		* Separate RTGs
	* Too complex
		* Better just use one proper designed unit in the penetrator
		* Might be a combinations of repeaters and a big antenna at the lander and probe
		* Will also work better if the probe doesn't descent in a straight line.

	* Also a combination of repeaters and tether could be used.
	* The repeater will also work better if the penetrator move to the side. Repeaters do not risk breaking like the tether.
	* If one repeater fails the data rate will drop by a factor 10 or more, so we have a little redundancy.
