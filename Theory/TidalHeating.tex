\autsubsection{Heating of Europa}{Lukas Christensen}
Europa moves around Jupiter in a tidally locked orbit with a period of $\SI{3.55}{days}$\cite{website:europaOrbitData}. The eccentricity of the orbit results in a variation in distance to Jupiter of $\SI{12e3}{km}$, causing a tidal flexing of the moon because of the change in gravitational pull. Since Europa is not a perfectly elastic solid some of the flexing is dissipated leading to a release of energy throughout the structure, the magnitude of which is estimated to correspond to $10-\SI{100}{\frac{mW}{m^2}}$ on the surface of the moon\cite{article:barr2014a}. Furthermore, radioactive decay within the core is estimated to provide an additional $6-\SI{8}{\frac{mW}{m^2}}$\cite{article:barr2014a}. These sources of heating mean that it should be possible for a liquid ocean to exist beneath the layer of ice that can be observed to cover Europa, although the exact configuration of such an ocean is still unknown\cite{article:barr2014a}.\\

\noindent
How exactly the tidal heating is distributed within the moon has a great impact on the structure and thickness of the ice sheet, and is a matter of some debate. Some argue that almost all of the energy is deposited in the core\cite{article:lowell2005a}, while others believe that a considerable amount is dissipated in the ice itself\cite{article:mccarthy2016a}. However, most agree that energy dissipated in the liquid ocean is of such low magnitude compared to the other sources that it can be disregarded\cite{article:chen2014a}. Not only does this affect the temperature profile of the ice, it may also impact the ice-water-interface at the bottom of the ice layer.\\

\noindent
If the vast majority of the energy is dissipated in the core, it should be fair to assume that the bottom of the ice will form a sharp interface with the ocean, similar to what is witnessed on terrestrial icebergs. The reasoning being that with the energy coming from below, the ice will slowly melt upwards until an equilibrium is reached. On the other hand, if a large amount of the energy is deposited in the ice itself, the interface between water and ice will likely be smeared out resulting in a section of partially melted slush. In this case, the argument is that the dissipation in the ice is lessened as it melts, meaning that part of the ice will be stuck between a solid and a liquid state in order to sustain the equilibrium.\\

\noindent
Whether or not this layer of slushy ice exists can potentially have large consequences of the mission. If potential life is not sustainable in partially frozen water, a penetrator mission will have to continue on through the slushy layer, down to the water level before measurements can be made, which raises the question of how to stop the descent once this point is reached. Otherwise, such a mission could simply anchor itself in the bottom of the ice, and perform its experiments at the interface. Communication through the slush layer will likely also be an issue, as the water will block radio communication while the suspended ice will block sound waves.\\

\noindent
Until such a time when the exact nature of the tidal heating is known, and by extension the nature of the ice layer, mission design must be performed with both contingencies in mind. Otherwise, the mission will have small chances of succeeding.