\autsection{Reaction Control System}{Søren Jeppesen}\label{chap:RCS}

The reaction control system (RCS) of the lander is the system of thrusters that control the attitude and provide minor translation control. The system uses a series of small engines to provide three-axis control of the lander to correct minor position and velocity errors, divergence from the optimal descent profile, and to to ensure that the rotational properties of the lander are ideal. The RCS is additionally to provide precise maneuvering during the landing stage to ensure safe and proper landing at the designated target.\\

This section will form an overview of the RCS envisioned for the Europa mission as well as some of its capabilites. A detailed schematic of the thrusters, fuel pump system, electrical power system and such will not be discussed. These design can instead be taken from already existing RCS designed for similar needs, and then modified to suit the Europa mission.\\

\subsection{Design}

The RCS envisioned for the Europa mission has been chosen from its similarities to lunar landing missions. As the Moons gravity is almost the same as Europas, lunar descent vehicles  will serve as good inspiration for the capabillity needs of the RCS. Particularly, the ESA Lunar Lander and Chang'e 3 lunar descent vehicles have been used to model the RCS for the Europa mission. In figure X are seen the RCS for these missions.\\


					INSERT PICTURES OF LUNAR RCS\\


Based on these missions, the RCS for the Europa mission will consist of 16 thrusters each providing 20 N of thrust. The thrusters are positioned in four clusters of four thrusters each, positioned around the center of gravity for the lander. The total thrust output this setup can generate is smaller than those of the Chang'e 3 and the Apollo Module, but this is off-set  by the much lower weight and the choice to not use the RCS to support he main engine in killing dV. Figure X shows a cluster as well as a single thruster belonging to the cluster.\\

					INSERT PICTURES OF THRUSTERS\\


The thrusters use the green propellant as opposed to the bi-propellant used by the main engines. The properties of the propellant is outlined in section XXXX, but in short, with a high specific impulse of XXX this propellant should be well suited for the RCS without the risk of contaminating the landing site. The choice of green propellant over bi-propellant increases the difficulty of designing the RCS as most common systems use a hydrazine propellant along with an oxidizer, thus limiting the already existing schematics that can be drawn on.\\

\subsection{Rotational capabilities}

Analyzing the rotational ability of the RCS is done using a simplifed model of the lander. In this model, the moment of inertia for the lander is presented as a solid frustum cone with a solid cylinder inside. The cone represents the lander and has a uniform density througout the volume of the cone. The cylinder inside represents the payload and has a uniform density that is different from that of the cone. Figure X shows the model used for representing the moment of inertia.\\

					INSERT PICTURE OF CONE + CYLINDER\\


The moment of inertia for this structure is\\

					INSERT MOMENT OF INERTIA\\

Where M1 is , M2 is , R is , and h is . Due to the symmetri of the lander in this model, the y and x directions have identical moment of inertia. For the lander, the values of these are M1 = , M2 = , R = and h = . This gives the lander a moment of inertia of I = . The angular acceleration in the three directions are then determined from\\ 
\begin{equation}
    I a = r \times F
\end{equation}
    
here, I is the moment of inertia, a is the angular acceleration, R is the position vector of the point where the force is applied, and F is the force vector. In the x and y directions, the RCS is capable of providing 40 N of thrust, and in the z direction the RCS can proviode 80 N of thrust. The angular acceleration that the RCS can deliver is then XXX in the x and y directions, and XXX in the z direction at the lower edge of the lander. This in turn makes the rotation period of the lander XXX in x and y direction, and XXX in the z direction.\\

To decrease the rotation period of the lander, the thrusters can instead be placed on beams stretching out from the lander a seen on figure X. Given 1 m long beams, the rotation period intead becomes XXX in the x and y direction, and XXX in the z direction.\\

					INSERT PICTURE OF MODEL WITH ARMS\\

\subsection{Velocity correction}

During the descent, the lander will likely be required to make small corrections to the to velocity in order to maintain optimal descent profile. With the suggested RCS, the lander will have two thrusters available for horizontal corrections with respect to the lander, and four thrusters available for vertical corrections with respect to the lander. With the lander sitting at 3986 kg of mass with a full fuel tank and each thruster providing 20 N of thrust, from Newton's second law of motion, the acceleration the RCS can provide horizontally is XXX, and the acceleration vertically is XXX. 


\subsection{Slide prevention}

In the final stage of the descent, the RCS can be used to prevent horizontal sliding (see figure X) from the lander. In the case where the lander maintains upright position, the RCS provides 40 N of thrust and assuming an almost empty fuel tank, weights around 1000 kg. The RCS is then able to provide XXX of horizontal acceleration.\\

					INSERT PICTURE OF SLIDING\\

Alternatively, the lander can turn 45 degrees and activate another two thrusters to counter sliding. At this angle, each of the four sliders will provide 14.14 N of thrust in the direction opposite the sliding, for a total of 56.56 N. The horizontal acceleration is then XXX. With XXX seconds required to turn this angle, using this method will be more effective if more than XXX seconds are required to counter sliding.\\

					INSERT PICTURE OF TURNED LANDER\\


\subsection{Soft landing capabilities}

During the landing, the RCS must be able to maneuver the lander with high precision. To this end, the lander has eight 20 N thrusters for altitude control, providing 80 N of thrust in either upwards or downwards direction, and eight thrusters for horizontal control providing 40 N of thrust in either of the four horizontal directions. The horizontal acceleration provided by the RCS is mentioned in the previous section to be XXX, and the vertical acceleration is XXX. This should be sufficiently low to provide fine adjustment for the landing in the gravitational environment of Europa.
