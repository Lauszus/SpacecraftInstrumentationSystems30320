\chapter{Appendix Optical camera}\label{app:majamatlab}

This is the matlab code used for calculating FOV and GSD of cameras used for optical navigation. The sensor used is sthe $\mu$ASC sensor. 

\begin{lstlisting}
%% Geometric optics considerations for FOV and GSD 
f=[0.003  0.005 0.007  0.009  0.011  0.013 0.015]; %Focal lenght
pix=8.6*10^(-6)*752; %detector size

viewangler=2*atan(pix./(2*(f)));%view angle in rad
viewangled=viewangler*180/pi; %view angle in degrees

for n=1:length(viewangler)
dist=[10 100 1000 10000 100000 200000];
for j=1:length(dist)

FOV(n,j)=(2*dist(j)*(tan(1/2*viewangler(n))));
end 
end 

figure
for n=1:length(viewangler)

semilogy(FOV(n,:))
xlabel('Altitude [m]' );
ylabel('Ground FOV [m]');
set(gca,'XTick',[1 2 3 4 5 6] ); %This are going to be the only values affected.
set(gca,'XTickLabel',[10^1 10^2 10^3 10^4 10^5 2*10^5] )
h=legend('3 mm', '5 mm', '7 mm', '9 mm', '11 mm', '13 mm', '15 mm','Location','northwest');
 %v = get(h,'title');
 %set(v,'string','Focal Length');
hold on
%
end 
print -dbmp16m 'FOV.png'
hold off
%%
%Find ground sampling distance in m/pix
figure
for n=1:length(viewangler)
    GSD= FOV(n,:)/(752/2); 
    semilogy(GSD)
    xlabel('Altitude [m]' );
ylabel('GSD [m/pixel]');
set(gca,'XTick',[1 2 3 4 5 6] ); %This are going to be the only values affected.
set(gca,'XTickLabel',[10^1 10^2 10^3 10^4 10^5 2*10^5] )
h=legend('3 mm', '5 mm', '7 mm', '9 mm', '11 mm', '13 mm', '15 mm','Location','northwest');
    hold on 
    
end
print -dbmp16m 'GSD.png'

\end{lstlisting}

% One can also include a script directly like so:
	% \lstinputlisting{path/script.m}