\autsection{Descent Profile}{Aaron Gornott}

To safely deploy the ice penetrator on Europas surface it is necessary to kill the delta v. Europas thin atmosphere of \begin{math}10^{-{12}}\end{math} bar is not sufficient to consider reasonable aero breaking. As a result we cannot utilize air drag and rely on a fully propulsive descent. The major disadvantage is that without the help of the environment we need to bring all the breaking fuel with us. A positive aspect is that our spacecraft design does not have to handle atmospheric entry heat and does not need aerodynamic shaping to account for turbulences.\\
\\
When the descent vehicle is released from the elliptical Jupiter orbit there are two general options to approach the surface. The first option is a direct descent where the delta v is killed in one or multiple burns (can include transfer orbits)  this is the most energy efficient profile because it only contains breaking burns. If transfer obits are utilized it can be more appropriate to call it semi direct descent. A transfer orbit is the most efficient way to lower the altitude without gaining too much velocity. The limit how deep a transfer orbit can be chosen depends on the power of the breaking thrusters. As soon as the spacecraft travels slower than orbital velocity there is the obvious need that is has to kill its remaining speed as well as the accumulated speed during is free fall towards the ground in this time constraints, if a soft touch down is intended. \\
The second landing profile option includes a parking orbit where the spacecraft can adjust its navigation instruments and prepare for the touchdown without time pressure. Navigation instruments benefit in this scenario from the lower surveying altitude that increases measurable terrain detail resolution. The gained time also advances safety because a final systems check can be performed. This adds flexibility for solving last minute problems . It is evident that those flexibilities come at a cost, a big disadvantage is the additional exposure to the high radiation environment around Europa (for more details see the radiation section). Another more general issue is that it requires propellant to enter and leave a parking orbit. Despite the mass issue the lower altitude limit has to be determined in regards of the stability of the orbit (interference forces are for example: gravity irregularities, radiation pressure or atmospheric drag). As stated before the atmospheric pressure is so low that the stability is not an issue for even for multiple month, the same is true for the radiation pressure. Due to the radiation a realistic parking orbit will be measures in minutes not month. An Europa environment feature that is not well understood for now are gravity irregularities caused by areas of different density in the moon. However it is possible to make some optimistic assumptions about a evenly distributed gravity field. The based on data from the Galileo probe the subsurface ocean could have a average depth from 80 to 170 km. All this water, that can be assumed distributed evenly, might outbalance gravity irregularities. Concluding from this consideration it can be possible to have a stable orbit as low as 10 km, the orbital speed at this altitude is 1427 m/s.\\
\\
If a parking orbit landing profile is chosen, for the benefit of the most efficient descent it makes sense to do a Hohmann transfer to the lowest possible point, since this reduces the energetic  worse accumulated delta v during the free fall. Regarding the Hohmann transfer it should be mentioned that it consist of two burns, first to get into transfer orbit and the second to get into the indented target orbit. Hohmann formulas assume a instant velocity change, this idealistic model does not hold, the reality is less efficient because actual rocket motors need time to apply the available thrust. Nevertheless future Hohmann calculation in this mission report will also adopt the idealized idea of instant velocity changes.\\
\\
Next phase in the landing profile is to leave orbit and approach the ground. To leave the orbit the breaking force need to be applied horizontal in regards to the orbited body. Finally approaching the ground however call for vertical thrust (after horizontal v is zero). The smooth transition from those thrust vectors during descent (the same rules apply also for rocket ascending)  is called gravity turn or zero-lift turn. While the breaking rockets decrease the horizontal component the gravity increases the vertical or nadir (towards orbited body) component. To balance these two vectors the spacecraft tilts its thrust vector towards zero angle of attack.\\
As optimal a gravity turn is from a energetic perspective it is not necessarily from a safety perspective. When finally touch down the lander on the surface it has to be ensured that no horizontal forces are existent that can cause the lander to flip over. From a safety perspective horizontal velocity should be reduced to zero with a sufficient enough time window to correct potential errors or avoid surface obstacles. Obstacle avoidance gains when translate maneuvers in all direction are equally possible. The very final phase before surface contact therefore should be a purely vertical descent. This touchdown burn however should be keep as short as possible and should be applies to the latest possible time. Vertical breaking long before touchdown results in a much longer descent time or at worst case includes some hovering periods. This entire descent time accumulates gravity acceleration and due to the fact that this fuel is needed in the very last descent phase makes it very precious. As described slowing down early increase the overall delta v demands, that worsens the fuel requirements, and propagates through the whole system. If a touchdown rocket motor can only apply a certain maximum thrust more fuel weight means it has to fire earlier as a consequence the delta v requirement is worsened again. As important it is to escape this fuel requirement spiral balanced and careful consideration have to be chosen in respect of landing safety. Final touchdown is likely the most critical phase at the entire mission that can lead to a mission failure.\\

\autsubsection{Mathematical Model}{Aaron Gornott}
on it...