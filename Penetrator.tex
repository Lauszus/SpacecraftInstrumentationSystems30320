\chapter{Penetrator}

* What kind of ice can we expect? Mud, silt etc?

\section{Drilling Methods}

\subsection{Mechanical Drilling}

\subsection{Chemical Drilling}

\subsection{Explosive Drilling}

\subsection{Sputtering}

\subsection{Laser Drilling}

\subsection{Light Concentration Drilling}

\subsection{Melting}

* Water transportation from tip to the end of the penetrator (ref section about convection flow)

* Should measure flow of the water - descent rate

* Water flow to the instruments. Will have to use a pump in order to increase the flow rate.

* Measure conductivity of the water.


\section{RTG design (Ricard)}
\subsection{Radiation from Pu238}
After looking at the different materials, Plutonium-238 (Pu238) is radioactive material that is going to be used for the mission due to the the half-life of 87.7 and the minimum radiation of gamma and neutrons. 

The probability of a alpha-decay from a Pu238-atom is close to 100\%: 
\begin{equation}
^{238}\text{Ph} \quad \rightarrow \quad ^{234}\text{U} \quad + \quad ^4 \text{He}
\end{equation}
The alpha particle is relatively big with two protons and two neutrons and is therefore easy to shield against. Even if the possibility of an alpha decay is close to almost certain, there is a small possibility of either a spontaneous fission or a cluster decay. A spontaneous fission happens for heavy atoms, where the atom split into two almost equal sized daughter atoms and a number of free neutrons. The possibility of a spontaneous fission for Pu238 is $1.9 \cdot 10^{-7}\%$, but with more than $10^{24}$ atoms for each kg of plutonium, it is certain to happen. The daughter atoms can vary. A cluster decay is something between an alpha decay and spontaneous decay where a cluster of protons and neutrons are emitted. For the Pu238 there is a $1.4 \cdot 10^{-14}\%$ chance that a Si32-atom is emitted leaving a Hg206-atom, and a $6 \cdot 10^{-15}\%$ chance that a Mg30-atom and a Mg28-atom is emitted leaving an Yb180-atom. The daughter isotopes are also radioactive and are most likely to emit a beta-particle. This particle is either an electron or a positron depending on the charge of the particle with a mass much smaller than neutron or proton but since it has an electric charge it can rather easily be stop by a shielding metal. 

A bigger problem is the free neutrons that for example are emitted due to a spontaneous fission.  The neutrons do not have any electric charge and therefore they will not be stopped by an electric field making them extremely penetrating. To stop a neutron is has to be absorbed by a nucleus or collide with a particle with same speed but opposite velocity and mass as the neutron. Since the materials used to shield alpha and beta particles, for example lead, has a rather small cross-section and hew nuclei per unit mass, hence the neutron can easily pass trough the shielding. Furthermore if the neutron is absorbed to the atom it is likely that the new atom is radioactive as well. Therefore it is important to stop the neutrons, because they can cause a lot of damage if they hit the instruments on board the penetrator. In order to do so protium, a hydrogen atom with no neutron, can be used to absorb the neutron creating in deuterium. Water has a high number density to mass ratio, and is hereby optimal to use as a shielding against the free neutrons. 


\subsection{Shielding}
As mentioned a typical shielding material is lead, but for this case zirconium would be a better choice, first of all because it has a melting temperature that is three times higher than lead, and second of all because the material is lighter. It is still not possible to stop the free neutrons, where we instead will use water. First the Plutonium will be surrounded by the zirconium, afterwards a layer of water and in the end some thermocouplers will generate electricity from the heat. 

\subsection{Heat generation}

To calculate how much energy one alpha decay will produce, the energy equation derived from Einstein is used:
\begin{equation}
\begin{aligned} 
\text{$E_{atom}$} & ={} mc^2 \\
& = (m_{Pu238} - [m_{U234} + m_{He4}]) \cdot c^2 \\
& = (238.049559894U - [234.040952088U + 4.00150646649U])\\
& \cdot 1.66053904020\cdot 10^{-27}\text{kg} \cdot 2.99\cdot 10^8 \text{m/s}\\
& = 1.06\cdot 10^{-12} J 
\end{aligned}
\end{equation}

When knowing how much one decay will produce of thermal energy, the number of decays in a radioactive material can be calculated as following:
\begin{equation}
A = A_0 \cdot 2^{\frac{t}{\tau}}
\end{equation}
where \textit{$A_0$} is the initial number of atoms in a lump of a radioactive material, \textit{A} is the atoms left in the material after a specific time, \textit{t}, with a half-life of  \textit{$\tau$}. The energy produced by one kg of Pu238 over a timespan of one second will then be: 
\begin{equation}
\begin{aligned} 
\text{$E_{kg}$} & ={} E_{atom} \cdot [A_0 - A] \\
& = E_{atom} \cdot \left[A_0 - A_0 \cdot 2^{\frac{t}{\tau}}\right] \\
& = E_{atom}  \cdot \left[\frac{1}{m_{Pu238}} - \frac{2^{\frac{t}{\tau}}}{m_{Pu238}} \right] \\
& = 671 J
\end{aligned}
\end{equation}

We here see that for each kg of Pu238, 671W energy is generated in form of thermal energy. Since the amount of Pu238 decreases for each decay, the generated energy will also decrease, but rather slowly since the half life is 87.7 years. 
After the decay to U234, the uranium is also radioactive, undergoing an alpha decay as well. The lifetime is more than 2400 years and therefore the decay from this isotope is much more protracted. The decay chain from Pu238 will end with the stable isotope Pb206 after 12 decays from either alpha or beta decay. \\

Since we need 2kW of thermal energy and around 100W as electrical energy for the instruments on the penetrator we would need around 5 kg of plutonium. Some extra plutonium is needed because the amount of radioactive material decreases with time, and since the mission takes a couple of years the heat generation have decreased a factor 
\begin{equation}
N = N_t \left(\frac{1}{2}\right)^{t/\tau} 
\end{equation}


With a density of 19816 kg$/m^3$, with a mass of 5 kg we would need a a plutonium rod that is 30 cm long and 1.65 cm in radius. It is a good idea that the plutonium has a high surface area compared to the volume, because the heat was to be distributed to the water running in the heat pipes(maybe). Furthermore it is a good thing that the plutonium has the same shape as the penetrator. (Explanation to come) 


\subsection{Heat pipes}
To distribute the heat that is generated from the plutonium. In order to do so, the water will be under high pressure. If the pressure inside the pipe system is 50 bar, or around 5 MPa, the water has a boiling temperature at $264^\circ$C. This will give some advantages for the system in the sense that it is almost impossible to maintain a temperature below $100^\circ$C which is waters normal boiling point, when transporting 2kW of heat. 

The pipe system will consist of pipes surrounding the RTG in the bottom and middle of the penetrator. Since the RTG is 30 cm long the pipes will be this as well with 18 tubes cut in half. The water will then flow out to the 18 tubes in the outer part of the RTG. Since most of the heat of generated should be directed downwards 12 of the tubes only cover the lowest $1/4$ of the penetrator. The other 6 tubes will go all the way to the top of the penetrator in order to ensure that the penetrator will not freeze in because the top of the penetrator is too cold. With an internal pressure of 5MPa, the thickness of the tubes that have water in them will be around 1 mm depending on the material if the other radius of the tubes are 1.48 cm. The heat pipe system will then weight around 6 kg. Furthermore the system will take up 6.5 liters of space. 

Insert image of top view of the pipe system and RTG. 
Insert image of a sideways look of the system.


\section{Selected Design}

* Sketch of overall design (use 3D models for the ice melting simulations)

\subsection{Melting through the ice} % Lukas, KSL

\subsection{RTG on Top}

\subsection{RTG on Bottom}

* How do we protect the rest of the instruments against the radiation?





\subsection{Thermal Design}

\subsection{Water Convection}

\subsection{Anchor Design}

\subsection{Anchoring and Deployment}

* Ref to composition of the ice and theory about lakes?

\subsection{Submarine}

\section{Mechanisms and Instrumentation}

\subsection{Navigation and Dirigibility}

\subsection{Detection of Depth and End of Ice Column}

* Echo sounder etc

\section{Communication Systems}

\subsection{Ice Losses}
As we expect, Europa's subsurface consists mainly of several kilometers of ice, which we want to penetrate with electromagnetic waves to establish the communication link between the penetrator and the lander vehicle. Fortunately, a lot of research has been made in the recent decades on how efficient an electromagnetic wave can penetrate different ice layers and on which parameters can affect this propagation. These principles are applied widely in subsurface radar sounding that take place in polar areas, but also in planetary exploration (Mars). For a wide range of frequencies, ranging from MHz to GHz, dielectric losses of ice are independent of frequency. By that it is meant that, the number of wavelengths, which can penetrate into ice before being attenuated to a given fraction of its initial amplitude ($1/e$ of initial amplitude) is approximately the same regardless of frequency. The above implies that the longer the wavelength, the deeper the radar signal can penetrate before being attenuated below the detection of our equipment. Thus, deep ice penetration requires that the radar operates at the lowest possible frequency. 

\paragraph{Dielectric properties of ice}
(For the following two paragraphs \cite{Kofman_2010} has used as main reference.)The permittivity $\epsilon$ of a material is a property describing how much more energy is stored though change separation than in vacuum. Frequency dependence of permittivity occurs because change separation does not happen instantaneously. Changes separate with finite velocities, thus if the external field is reversing polarity too quickly the changes cannot move fast enough to keep up. The frequency at which the charges fully separate and are in constant motion is called the relaxation frequency. At frequencies below the relaxation frequency the permittivity plateaus at the low frequency limit (static) $\epsilon_{s}$, which is often call dielectric constant. At high frequencies above the relaxation frequency the permittivity plateaus at the high frequency limit $\epsilon_{\infty}$. Moreover, ice crystal formation has an impact on polarization, which primarily depends on temperature.

Debye model takes into account the above mentioned theory about dielectric constant of ice and is described by the following equations: 

\begin{equation}
    \epsilon=\epsilon_{\infty}+\Delta \epsilon \frac{\Delta \epsilon}{1+j \omega \tau}
    \label{dielectric}
\end{equation}

, where $\omega = 2\pi f$, $\Delta \epsilon=\epsilon_{s} -\epsilon_{\infty}$ and $\tau$ is the dielectric relaxation time. \\
The permittivity is a complex function of frequency and usually is described by its real and imaginary part.

\begin{equation}
    Re (\epsilon)=\epsilon_{\infty}+\Delta \epsilon \frac{\Delta \epsilon}{1+ \omega^2 \tau^2}
    \label{real}
\end{equation}

\begin{equation}
    Im (\epsilon)=\frac{\Delta \epsilon\ \omega\  \tau}{1+ \omega^2 \tau^2}
    \label{imag}
\end{equation}
The loss tangent (tan $\delta$) is defined by the ratio of these two parts and characterizes the attenuation of the electromagnetic waves in a medium due to ohmic conductivity $\sigma$. 

\begin{equation}
    tan \delta=\frac{Re(\epsilon)}{Im(\epsilon)}=\frac{\sigma}{\omega\ Re(\epsilon)}
    \label{tan}
\end{equation}
The conductivity $\sigma$ of the medium is directly proportional to the imaginary part of the dielectric constant.

Because of the complexity of equations (\ref{dielectric} - \ref{tan}) an approximated expression has been developed to compute the attenuation.

\begin{multline}
    a=0.129 \sqrt{Re(\epsilon)}\ f (\sqrt{1+tan^2 \delta}-1)^{1/2} \approx \\
    \approx 0.091 \sqrt{Re(\epsilon)}\ f\ tan \delta \approx 0.0009\ \sigma\ dB/m
    \label{losses}
\end{multline}
where, $\sigma$ is in $\mu S m^{-1}$. As one can see from eq. (\ref{losses}), attenuation's value is directly proportional to frequency, or in other words to the conductivity of the medium. Additionally, the static dielectric constant of pure ice is heavily depended on the orientation of electric field with respect to the crystal's axis. The effect of pressure is about 1\% per kbar for polycrystalline ice \cite{Kofman_2010}. The above formulas and their approximations concerning the electromagnetic waves can be used adequately for very low temperatures, as the ranges we are interested.

Nevertheless, the losses due to pure ice are well documented in bibliography, there is a big gap regarding ice impurities on icy moons. The absence of these data are due to uncertainties and lack of knowledge of the physical nature of impurities on these satellites. This problem was studied by  Moore \cite{Moore_2000} and Chyba \cite{Chyba_1998} for Europa. Moore considered three types of water ice on Europa, produced by three basic processes occurring on Earth. The first one is meteoric ice formed by atmospheric precipitations, sea ice formed by the freezing of water close to the atmospheric interface and marine ice forming beneath ice shelves directly from ocean water. Moore concluded that similar processes are likely to occur on Europa as well, and that the most probable form of ice would be marine ice. In figure \ref{impurities} Moore sums up the attenuation from different type of impurities in ice \cite{Moore_2000}.

\begin{figure}[ht]
\centering
\label{impurities}
\includegraphics[width=1\textwidth]{figures/Moore2.jpg}
\caption{Attenuation, $a$, is in dB/km at 251 K and corresponds to the one way attenuation due to  ice impurities, in case of a sounding radar. Columns I, II, and III are computed one-way attenuation, in dB/km, for ice shells with base temperatures of 270, 260, and 250 K, respectively. The range of values for each of these corresponds to surface temperatures of 50 and 100 K. These values are independent of shell thickness since the temperature profile is stretched to the ice thickness. The M column represents the plausibility of the ice type for Europa; 0 is least likely while 3 is more likely, given the present understanding
of Europa. The distribution coefficients $k_{0}$ and $k_{MI}$ affecting the marine ice models come from laboratory experiments \cite{Moore_2000}. \textbf{+ appendix}}
\end{figure}
The above calculations data are not taking into account a possible scattering mechanism of electromagnetic wave due to ice layers that can exist in the crust. The scattering effect has a significant impact on the attenuation level and depends strongly on the dimensions of cavities in the medium compared to the wavelength. The two main mechanisms of scattering coming from the ice crust are surface scattering and scattering by volume irregularities. Both these effects can change considerably the penetration depth of the wave into the ice, but also the ratio of any subsurface echo to surface clutter. As we can understand the scattering depends strongly on the wavelength and surface parameters of the body under investigation. 

The conclusion is that the expected one way attenuation because of impurities of ice is in the range 1-8 dB/Km and this number is independent of the frequency. Although, the frequency dependence of attenuation due to scattering mechanisms dictates the use of as low frequencies as possible in order to achieve a deep penetration. The main bottlenecks in that case are two. The first one is that the choice of frequency has an influence on the characteristics of instrumentation and especially on the size of the antenna and the second one is Jupiter's radio emissions spectrum. Figure \ref{J_spec} depicts how Jupiter's activity affect the electromagnetic environment of its moons. Clearly can be seen that frequencies from almost zero Hz up to 50 MHz are dominated from Jupiter's radio emissions. Thus, the exact choice of frequency results in a trade off between science requirements and technical limitations taking into account also the physical constraints of the environment under research. 

\begin{figure}[ht]
\centering
\label{J_spec}
\includegraphics[width=0.7\textwidth]{figures/below100.jpg}
\caption{Jupiter radio spectrum based on Cassini-RPWS data,
normalized to a distance of 1 AU. Green curve: rotation averaged
emission. Blue curve: rotation averaged emission at times of intense
activity. Red curve: peak intensities during active periods. \cite{Grie_meier_2005}}
\end{figure}


\begin{figure}[ht]
\begin{center}
\includegraphics[width=0.7\columnwidth]{figures/below100}
\caption{Replace this text with your caption%
}
\end{center}
\end{figure}

\subsection{Communication to Lander}

\subsection{High Directivity Link}

\subsection{Relay Systems}
