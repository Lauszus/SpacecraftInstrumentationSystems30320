\chapter{Penetrator}

* What kind of ice can we expect? Mud, silt etc?

\section{Drilling Methods}

\subsection{Mechanical Drilling}

\subsection{Chemical Drilling}

\subsection{Explosive Drilling}

\subsection{Sputtering}

\subsection{Laser Drilling}

\subsection{Light Concentration Drilling}

\subsection{Melting}

* Water transportation from tip to the end of the penetrator (ref section about convection flow)

* Should measure flow of the water - descent rate

* Water flow to the instruments. Will have to use a pump in order to increase the flow rate.

* Measure conductivity of the water.

\section{Selected Design}

* Sketch of overall design (use 3D models for the ice melting simulations)

\subsection{Melting through the ice} % Lukas, KSL

\subsection{RTG on Top}

\subsection{RTG on Bottom}

* How do we protect the rest of the instruments against the radiation?

\subsection{RTG design}

* Can we make it smaller?

* Thermal design

\subsection{Thermal Design}

\subsection{Water Convection}

\subsection{Anchor Design}

\subsection{Anchoring and Deployment}

* Ref to composition of the ice and theory about lakes?

\subsection{Submarine}

\section{Mechanisms and Instrumentation}

\subsection{Navigation and Dirigibility}

\subsection{Detection of Depth and End of Ice Column}

* Echo sounder etc

\section{Communication Systems}

\subsection{Ice Losses}
As we expect, Europa's subsurface consists mainly of several kilometers of ice, which we want to penetrate with electromagnetic waves to establish the communication link between the penetrator and the lander vehicle. Fortunately, a lot of research has been made in the recent decades on how efficient an electromagnetic wave can penetrate different ice layers and on which parameters can affect this propagation. These principles are applied widely in subsurface radar sounding that take place in polar areas, but also in planetary exploration (Mars). For a wide range of frequencies, ranging from MHz to GHz, dielectric losses of ice are independent of frequency. By that it is meant that, the number of wavelengths, which can penetrate into ice before being attenuated to a given fraction of its initial amplitude ($1/e$ of initial amplitude) is approximately the same regardless of frequency. The above implies that the longer the wavelength, the deeper the radar signal can penetrate before being attenuated below the detection of our equipment. Thus, deep ice penetration requires that the radar operates at the lowest possible frequency. 

\paragraph{Dielectric properties of ice}
(For the following two paragraphs \cite{Kofman_2010} has used as main reference.)The permittivity $\epsilon$ of a material is a property describing how much more energy is stored though change separation than in vacuum. Frequency dependence of permittivity occurs because change separation does not happen instantaneously. Changes separate with finite velocities, thus if the external field is reversing polarity too quickly the changes cannot move fast enough to keep up. The frequency at which the charges fully separate and are in constant motion is called the relaxation frequency. At frequencies below the relaxation frequency the permittivity plateaus at the low frequency limit (static) $\epsilon_{s}$, which is often call dielectric constant. At high frequencies above the relaxation frequency the permittivity plateaus at the high frequency limit $\epsilon_{\infty}$. Moreover, ice crystal formation has an impact on polarization, which primarily depends on temperature.

Debye model takes into account the above mentioned theory about dielectric constant of ice and is described by the following equations: 

\begin{equation}
    \epsilon=\epsilon_{\infty}+\Delta \epsilon \frac{\Delta \epsilon}{1+j \omega \tau}
    \label{dielectric}
\end{equation}

, where $\omega = 2\pi f$, $\Delta \epsilon=\epsilon_{s} -\epsilon_{\infty}$ and $\tau$ is the dielectric relaxation time. \\
The permittivity is a complex function of frequency and usually is described by its real and imaginary part.

\begin{equation}
    Re (\epsilon)=\epsilon_{\infty}+\Delta \epsilon \frac{\Delta \epsilon}{1+ \omega^2 \tau^2}
    \label{real}
\end{equation}

\begin{equation}
    Im (\epsilon)=\frac{\Delta \epsilon\ \omega\  \tau}{1+ \omega^2 \tau^2}
    \label{imag}
\end{equation}
The loss tangent (tan $\delta$) is defined by the ratio of these two parts and characterizes the attenuation of the electromagnetic waves in a medium due to ohmic conductivity $\sigma$. 

\begin{equation}
    tan \delta=\frac{Re(\epsilon)}{Im(\epsilon)}=\frac{\sigma}{\omega\ Re(\epsilon)}
    \label{tan}
\end{equation}
The conductivity $\sigma$ of the medium is directly proportional to the imaginary part of the dielectric constant.

Because of the complexity of equations (\ref{dielectric} - \ref{tan}) an approximated expression has been developed to compute the attenuation.

\begin{multline}
    a=0.129 \sqrt{Re(\epsilon)}\ f (\sqrt{1+tan^2 \delta}-1)^{1/2} \approx \\
    \approx 0.091 \sqrt{Re(\epsilon)}\ f\ tan \delta \approx 0.0009\ \sigma\ dB/m
    \label{losses}
\end{multline}
where, $\sigma$ is in $\mu S m^{-1}$. As one can see from eq. (\ref{losses}), attenuation's value is directly proportional to frequency, or in other words to the conductivity of the medium. Additionally, the static dielectric constant of pure ice is heavily depended on the orientation of electric field with respect to the crystal's axis. The effect of pressure is about 1\% per kbar for polycrystalline ice \cite{Kofman_2010}. The above formulas and their approximations concerning the electromagnetic waves can be used adequately for very low temperatures, as the ranges we are interested.

Nevertheless, the losses due to pure ice are well documented in bibliography, there is a big gap regarding ice impurities on icy moons. The absence of these data are due to uncertainties and lack of knowledge of the physical nature of impurities on these satellites. This problem was studied by  Moore \cite{Moore_2000} and Chyba \cite{Chyba_1998} for Europa. Moore considered three types of water ice on Europa, produced by three basic processes occurring on Earth. The first one is meteoric ice formed by atmospheric precipitations, sea ice formed by the freezing of water close to the atmospheric interface and marine ice forming beneath ice shelves directly from ocean water. Moore concluded that similar processes are likely to occur on Europa as well, and that the most probable form of ice would be marine ice. In figure \ref{impurities} Moore sums up the attenuation from different type of impurities in ice \cite{Moore_2000}.

\begin{figure}[ht]
\centering
\label{impurities}
\includegraphics[width=1\textwidth]{figures/Moore2.jpg}
\caption{Attenuation, $a$, is in dB/km at 251 K and corresponds to the one way attenuation due to  ice impurities, in case of a sounding radar. Columns I, II, and III are computed one-way attenuation, in dB/km, for ice shells with base temperatures of 270, 260, and 250 K, respectively. The range of values for each of these corresponds to surface temperatures of 50 and 100 K. These values are independent of shell thickness since the temperature profile is stretched to the ice thickness. The M column represents the plausibility of the ice type for Europa; 0 is least likely while 3 is more likely, given the present understanding
of Europa. The distribution coefficients $k_{0}$ and $k_{MI}$ affecting the marine ice models come from laboratory experiments \cite{Moore_2000}. \textbf{+ appendix}}
\end{figure}
The above calculations data are not taking into account a possible scattering mechanism of electromagnetic wave due to ice layers that can exist in the crust. The scattering effect has a significant impact on the attenuation level and depends strongly on the dimensions of cavities in the medium compared to the wavelength. The two main mechanisms of scattering coming from the ice crust are surface scattering and scattering by volume irregularities. Both these effects can change considerably the penetration depth of the wave into the ice, but also the ratio of any subsurface echo to surface clutter. As we can understand the scattering depends strongly on the wavelength and surface parameters of the body under investigation. 

The conclusion is that the expected one way attenuation because of impurities of ice is in the range 1-8 dB/Km and this number is independent of the frequency. Although, the frequency dependence of attenuation due to scattering mechanisms dictates the use of as low frequencies as possible in order to achieve a deep penetration. The main bottlenecks in that case are two. The first one is that the choice of frequency has an influence on the characteristics of instrumentation and especially on the size of the antenna and the second one is Jupiter's radio emissions spectrum. Figure \ref{J_spec} depicts how Jupiter's activity affect the electromagnetic environment of its moons. Clearly can be seen that frequencies from almost zero Hz up to 50 MHz are dominated from Jupiter's radio emissions. Thus, the exact choice of frequency results in a trade off between science requirements and technical limitations taking into account also the physical constraints of the environment under research. 

\begin{figure}[ht]
\centering
\label{J_spec}
\includegraphics[width=0.7\textwidth]{figures/below100.jpg}
\caption{Jupiter radio spectrum based on Cassini-RPWS data,
normalized to a distance of 1 AU. Green curve: rotation averaged
emission. Blue curve: rotation averaged emission at times of intense
activity. Red curve: peak intensities during active periods. \cite{Grie_meier_2005}}
\end{figure}


\begin{figure}[ht]
\begin{center}
\includegraphics[width=0.7\columnwidth]{figures/below100}
\caption{Replace this text with your caption%
}
\end{center}
\end{figure}

\subsection{Communication to Lander}

\subsection{High Directivity Link}

\subsection{Relay Systems}
