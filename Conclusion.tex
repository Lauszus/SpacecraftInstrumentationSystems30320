\chapter{Conclusion}
%Feel free to expand on these
\section{Theory}
\autsubsection{Ice Characteristics}{Lukas Christensen} 
Much is still unknown about the nature of Europa which means it is hard to definitively conclude much about the moon. What is known is that a large amount of energy is deposited through tidal effects which means that it is very likely that a liquid ocean exists beneath the ice. Unfortunately, the thickness of the ice and the nature of its bottom is still very much uncertain. However, it has been possible to achieve realistic estimates of the temperature profile using a combination of basic physics and the work done by other research groups, and these indicate that a lander + penetrator mission is realistically achievable. 

\autsubsection{Convection}{Lukas Christensen}
The theory behind convection has been analyzed and CFD simulations have been used to evaluate the use of natural convection for water transportation within a penetrator probe. It has been found that, even under ideal circumstances, the short lengths of realistic thermal probes combined with the low gravity of Europa result in too low a flow rate for it to be useful. It has therefore been concluded that an active pumping system is needed instead to provide the water flow to the instruments.

\autsection{Orbiter Imaging System}{Johannes Linde}
An imaging system, providing surface mapping, navigation imaging and surveying has been proposed. The system uses a dual camera structure, with both a narrow angle camera (NAC) and a wide angle camera (WAC). This choice is very flexible and will make it possible to perform both the primary imaging objectives (surface mapping) and secondary imaging objectives (orbit determination, navigation and lander guidance) with high performance. The WAC provides a ground resolution of 100 m/pix at 2000 KM whereas the NAC provides a ground resolution of 0.5 m/pix at 100 KM. This should be adequate for the landing site selection. As two cameras are provided, they can substitute each other to a degree, providing built in redundancy. If one camera fails, the mission can still continue but at a lower performance.

Enhancing the imaging system with imaging spectroscopy capabilities makes it possible to provide spectral data of the Europa surface. Different systems has been investigated and the most suitable, the wedge spectrometer has been selected for the final design proposal. When imaging spectroscopy is used to assess the surface material composition, the near IR and IR ranges are often essential. As the sensor is not capturing in this region, it limits the usability of the proposed spectrometer. While the system is still usable for an ice surface assessment, it is suggested that different sensors are considered for a future design to expand the spectral data into the IR domain.

The flyby duration has been estimated to 32 s. For each flyby, the WAC will generate 711.17 MB data, covering 6.5\% of the surface. This early estimate does not use spacecraft attitude control to provide additional coverage. For the NAC, each flyby will generate 7.1 GB of high resolution images, accounting for 0.65\% of the surface area. These estimates does not include the spectral data. Five flybys will be necessary to provide the required 30\% coverage. During all flybys, close to 3.65 GB of low resolution imaging will be generated and must be transferred to the ground station, before landing on the surface.

In the design proposal, the spectrometer will only be operating at ground resolution ranges of 5m/pixel to 50m/pixel, as higher resolutions will not be useful for the mission. Each flyby will generate 65.06 GB of spectral data, a fairly large amount when compared to non spectral imaging. However, only the regions of interest (ROIs) will be investigated further by the ground station, reducing the transferred data significantly. The remaining data can be transferred whenever a communication window is available.

If all imaging and spectral data are stored, a total of 360 GB (raw data) must be stored on the orbiter,  excluding the buffered data from the lander and penetrator. This may be too much to transfer, even during the total mission lifetime. Therefore, it will be necessary to filter and compress the data before transmission and these methods should be analysed for a future design.

\autsection{Communication}{Rasmus Lundby Pedersen, Gustavo Feijóo Carrillo, Ioannis Nissopoulos, and Kristian Sloth Lauszus}

Three communication links have been studied under the characteristics of an interplanetary mission. A deep space link for telemetry/telecommand and science data download using the facilities provided by ESA's ESTRACK network, including as well the tracking capabilities for accurate manoeuvring of deep-space spacecraft. X-band and S-band readily available transponder are used. Then a relay from Europa's surface lander and orbiter is used, consisting of multiple crossed half-wave dipole antennas placed on the surface module, and two similar antennas on the orbiter. The system will utilize the orbiters medium/low-gain amplifiers while an amplifier will be located in the surface module, which is embedded a few meters in the ice of Europa. This will create a reliable link between the surface module and the orbiting satellite during the short available communication window.\\

\noindent
Then with a more critical sense, ice embedded communications were studied, finding that temperature and impurity levels have great effect in performance of antenna radiation patterns, gain and radiation efficiency. A solution with stand-alone transceiver embedded in the ice as the penetrator makes its descent was ruled out for complexity, bulk and power source (RTG) procurement. And, in stead a mix of high-gain antenna on lander with a medium-gain and wide-beamwidth is used in the penetrator. This, in combination with available power from the penetrator RTG allow for a robust link through ice.

\autsection{Melting of preliminary hole}{Maja Tomicic}
Much further work should be done to ensure the feasibility of using a special rocket motor to melt a preliminary hole of $\sim$2 m in the ice. It has shown to be an ideal way of getting the payload out of the radiation environment. However, the amount of propellant needed is substantial and the efficiency of the rocket plume for this purpose in vacuum conditions is currently inconclusive.

\autsection{Landing navigation system}{Maja Tomicic}
A redundant autonomous navigation system has been designed, thus ensuring a pin-point landing on the desired landing spot. Initially, the navigation system uses one of the CHU's as a surface camera together with a database of known surface features (craters, cracks and ridges), two CHU's as star trackers and the integrated IMU system $\mu$IRU. The surface camera is designed to optimize the FOV and resolution for this purpose. This ensures direct position and attitude determination in an Europa fixed frame of reference.
As the lander approaches the moon surface fewer features in the surface camera FOV can be matched to the database. Therefore the relative velocity navigation system is included which uses the optical flow from unknown surface features to estimate the lander velocity with respect to the lunar surface. This is then integrated to give the position. At altitudes of about 50-100 meters from the surface the laser ranger is used to determine a precise height of the spacecraft relative to the surface. This information is used as hazard avoidance as to avoid deep gorges or steep hills. Finally, as the lander gets the intended landing area in sight, it will navigate relative to a hazard-free landing spot using the last CHU as a surface camera, with optimized optics, together with a laser as a structured light system. 

\section{Penetrator}
\autsubsection{Drilling Methods}{Lukas Christensen}
Several possible penetration methods have examined including mechanical, explosive, sputtering, laser, and thermal drilling. Based on thorough analyses of each possibility, thermal drilling has been found to be the most viable solution because of its simplicity and efficiency. To estimate the penetration time, a simulation of the drilling process has been designed and implemented, with the result of a penetration time of $\approx \SI{100}{\frac{days}{km}}$ for a 20 cm diameter penetrator with a 2 kW heating element. This result has been verified using generally accepted approximated equations, indicating that this result is realistic. Furthermore, issues associated with thermal drilling have been identified and possible solutions have been outlined.

\autsubsection{Penetrator vessel}{Paul Connetable}
A penetrator vessel able to drill through 10 km of ice on Europa and anchor there to take and analyze samples has been designed. After having modeled the pressure profile varying with depth on Europa, we found the pressure conditions the vessel would have to survive at 10 km depth. A careful choice of the material has led us use a very resilient titanium alloy, which is also resistant to corrosion. We then calculated the thickness and the total mass of the titanium shell the penetrator has, which are respectively of 2.9mm and 12.9 kg.
The penetrator has its own source of energy, under the form of an RTG. After some research, we decided to use some Plutonium-238 dioxide as the radioactive source, since it has both the right half-life and a good power production. The RTG would be a cylinder with a radius of 1.96 cm  and a height of 30 cm, containing small holes where water could flow. The RTG will be shielded with zirconium, and the heat will be transferred to a water circuit, which will circulate in the bottom of the penetrator in order to melt the ice. In the penetrator, the water will be contained in silver tubes, and the interface between these heat pipes and the ice will be made in gold.
The RTG will also be used to transform some of the heat energy in electric energy for all the electronics and the instruments on-board.
In order to stop at the right time and to dodge potential obstacles, the penetrator vessel will also ambark a small  radar and some antennas on its side. This radar will provide information on the ice column. It is then possible to couple the information obtained from this radar with the heat distribution system, in order to dodge potential obstacles. Indeed, by increasing artificially the heat distribution on one side of the penetrator, the ice on this side  will melt faster and the penetrator will slide on this side.
Finally, the penetrator will anchor itself at the bottom of the ice column, at the top of the ocean. The system used for anchoring will most likely be hooks and screws shot in the ice column, but choosing the right system can only be done with further information on the ice nature at this depth.

\autsubsection{Submarines}{Paul Connetable}
Smaller spherical or cylindrical submarines can also be done and taken in the penetrator vessel, in order to go deeper in the ocean, and possibly go at the bottom of the ocean. Made in the same titanium alloy as the penetrator vessel, they can be very interesting for sending small instruments in places where life could most likely be on Europa.

\section{Instrument Suite}
\autsubsection{Gas detection}{Agge Winther}
The idea behind bringing a gas sensor to help the search for life is discussed. Due to the conditions on Europa a whole new approach to gas detection must be set in place. The VEGA-instrumetent is then designed, and the theory of how it works discussed. Overall the instrument will be to much help in finding life and the overall originality of the instrument gives it good chances of detection gasses in a new way, dedicated for the conditions on Europa.

\autsubsection{pH \& Salinity Sensor}{Lukas Christensen}
The basic theory of pH and salinity sensors have been described, and the initial design of a possible instrument has been introduced. The use of ISFETs for the sensing elements seem preferable as these are very robust and well tested, but they do require frequent calibration which could prove to be an issue. 

\autsubsection{General conclusion}{Agge Winther}
The instrument suite is boiled down to the instruments just discussed, and with the definition on what type of life forms expected to be found on Europa, the Suite should be able to detect it. Most of the instruments have been costume designed for this mission, to make the work under the though condition Europas oceans offer, and at the same time considerations on power, mass and volume has been kept in mind. Overall the instrument suite is the most compressed version of a lab for finde the life expected to be in the oceans of Europa, and to work in its hostile environment.
