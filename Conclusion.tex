\chapter{Conclusion}

%* Tied to Problem Formulation
\autsection{Orbital Mechanics}{Alpha Beta}
Please write here and don't forget to change the author name! :)

\autsection{Orbiter Imaging System}{Johannes Linde}
An imaging system, providing surface mapping, navigation imaging and surveying has been proposed. The system uses a dual camera structure, with both a narrow angle camera (NAC) and a wide angle camera (WAC). This choice is very flexible and will make it possible to perform both the primary imaging objectives (surface mapping) and secondary imaging objectives (orbit determination, navigation and lander guidance) with high performance. The WAC provides a ground resolution of 100 m/pix at 2000 KM whereas the NAC provides a ground resolution of 0.5 m/pix at 100 KM. This should be adequate for the landing site selection. As two cameras are provided, they can substitute each other to a degree, providing built in redundancy. If one camera fails, the mission can still continue but at a lower performance.

Enhancing the imaging system with imaging spectroscopy capabilities makes it possible to provide spectral data of the Europa surface. Different systems has been investigated and the most suitable, the wedge spectrometer has been selected for the final design proposal. When imaging spectroscopy is used to assess the surface material composition, the near IR and IR ranges are often essential. As the sensor is not capturing in this region, it limits the usability of the proposed spectrometer. While the system is still usable for an ice surface assessment, it is suggested that different sensors are considered for a future design to expand the spectral data into the IR domain.

The flyby duration has been estimated to 32 s. For each flyby, the WAC will generate 711.17 MB data, covering 6.5\% of the surface. This early estimate does not use spacecraft attitude control to provide additional coverage. For the NAC, each flyby will generate 7.1 GB of high resolution images, accounting for 0.65\% of the surface area. These estimates does not include the spectral data. Five flybys will be necessary to provide the required 30\% coverage. During all flybys, close to 3.65 GB of low resolution imaging will be generated and must be transferred to the ground station, before landing on the surface.

In the design proposal, the spectrometer will only be operating at ground resolution ranges of 5m/pixel to 50m/pixel, as higher resolutions will not be useful for the mission. Each flyby will generate 65.06 GB of spectral data, a fairly large amount when compared to non spectral imaging. However, only the regions of interest (ROIs) will be investigated further by the ground station, reducing the transferred data significantly. The remaining data can be transferred whenever a communication window is available. 

If all imaging and spectral data are stored, a total of 360 GB (raw data) must be stored on the orbiter,  excluding the buffered data from the lander and penetrator. This may be too much to transfer, even during the total mission lifetime. Therefore, it will be necessary to filter and compress the data before transmission and these methods should be analysed for a future design.
\autsection{Landing navigation system}{Maja Tomicic}
A redundant autonomous navigation system has been designed, thus ensuring a pin-point landing on the desired landing spot. Initially, the navigation system uses one of the CHU's as a surface camera together with a database of known surface features (craters, cracks and ridges), two CHU's as star trackers and the integrated IMU system $\mu$IRU. The surface camera is designed to optimize the FOV and resolution for this purpose. This ensures direct position and attitude determination in an Europa fixed frame of reference.
As the lander approaches the moon surface fewer features in the surface camera FOV can be matched to the database. Therefore the relative velocity navigation system is included which uses the optical flow from unknown surface features to estimate the lander velocity with respect to the lunar surface. This is then integrated to give the position. At altitudes of about 50-100 meters from the surface the laser ranger is used to determine a precise height of the spacecraft relative to the surface. This information is used as hazard avoidance as to avoid deep gorges or steep hills. Finally, as the lander gets the intended landing area in sight, it will navigate relative to a hazard-free landing spot using the last CHU as a surface camera, with optimized optics, together with a laser as a structured light system. 

\autsection{Melting of preliminary hole}{Maja Tomicic}
Much further work should be done to ensure the feasibility of using a special rocket motor to melt a preliminary hole of $\sim$2 m in the ice. It has shown to be an ideal way of getting the payload out of the radiation environment. However, the amount of propellant needed is substantial and the efficiency of the rocket plume for this purpose in vacuum conditions is currently inconclusive.

%Feel free to expand on these
\section{Theory}
\autsubsection{Ice Characteristics}{Lukas Christensen} 
Much is still unknown about the nature of Europa which means it is hard to definitively conclude much about the moon. A large amount of energy is deposited through tidal effects which means that it is very likely that a liquid ocean exists beneath the ice. Unfortunately, the thickness of the ice and the nature of its bottom is still very much uncertain. However, it has been possible to achieve realistic estimates of the temperature profile using a combination of basic physics and the work done by other research groups, and these indicate that a penetrator mission is realistically achievable. 

\autsubsection{Convection}{Lukas Christensen}
The theory behind convection has been analyzed and CFD simulations have been used to evaluate the use of natural convection for water transportation within a penetrator probe. It has been found that, even under ideal circumstances, the short lengths of realistic thermal probes combined with the low gravity of Europa result in too low a flow rate for it to be useful. It has therefore been concluded that an active pumping system is needed instead to provide the water flow to the instruments.

\section{Penetrator}
\autsubsection{Drilling Methods}{Lukas Christensen}
Several possible penetration methods have examined including mechanical, chemical, explosive, sputtering, laser, and thermal drilling. Based on thorough analyses of each possibility, thermal drilling has been found to be the most viable solution because of its simplicity and efficiency. To estimate the penetration time a simulation of the drilling process has been designed and implemented, with the result of a penetration time of $\approx \SI{100}{\frac{days}{km}}$ for a 20 cm diameter penetrator with a 2 kW heating element. This result has been verified using generally accepted approximated equations, indicating that this result is realistic. Furthermore, issues associated with thermal drilling have been identified and possible solutions have been outlined.

\section{Instrument Suite}
\autsubsection{pH \& Salinity Sensor}{Lukas Christensen}
The basic theory of pH and salinity sensors have been described, and the initial design of a possible instrument has been introduced. The use of ISFETs for the sensing elements seem preferable as these are very robust and well tested, but they do require frequent calibration which could prove to be an issue. 


