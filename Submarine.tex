\autsubsection{Submarine}{Paul Connetable}

The penetrator will be anchored at the bottom of the ice crust, on the top of Europa's ocean, in order to find life. Still, if there is indeed life on Europa, there is a very high probability to find it and evidence of it at the bottom of the ocean. Any dead form of life would eventually sink at the bottom of this ocean, creating there a layer of sediments, mud and and thus, nutrients. The friction of water on the rocky core could also increase the water temperature near the bottom, creating there another very likely place for life development. This is why being able to send a probe, even small, to the bottom of Europa's ocean would add a crucial tool for finding life, or clues indicating the presence of life.


From another point of view, any further information found about Europa, even if it is not life related. It is for example of uttermost importance to understand better the physics and the system functioning to obtain measurements of the water flow, temperature, salinity at different depths, and to get water samples distant from the penetrator position.


To execute these missions, the best suited tool is a little submarine. This little submarine would be stored inside the penetrator, and released when the penetrator  reaches the ocean. These submarines would be made in titanium, and could contain any combination of instruments wanted. They would be some measurement platforms designed for smaller diverse missions, thanks to their versatility. They would communicate the results or the data to the penetrator either by direct liaison with it thanks to an optical fiber, or by sonar, or by bluetooth (for smaller ranges). In the case of sonar use, the penetrator could receive the signal thanks to the antenna placed at its bottom, which is used during the descent.


Two different kind of submarines are described here: spherical and cylindrical submarines.

\subsubsection{Spherical submarines}

\subsubsection{Cylindrical submarines}