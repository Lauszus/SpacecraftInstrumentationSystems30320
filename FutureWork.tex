\chapter{Future Work}

* Hard to test due to time limits, limited funds etc

    * Instruments are expensive and might be hard to get (fx RTG)

\autsection{RTG design}{Ricard Grove}
Normally a RTG is only used to generate electricity using thermocouples with a nuclear heat source to get a high temperature difference. All the thermal energy that are not used to generate electrical energy is not wanted and the RTG's are therefore designed to be able to get rid of as much heat as possible to i.e. prevent overheating. For this mission all the heat is wanted to melt though the ice and therefore a completely new design of the RTG is needed in order to tranfer the heat down to the tip of the penetrator. A preliminary proposal to the design can be seen in section \ref{sec:rtg}. Calculations of the heating and cooling of the water have been made using some simplifying assumptions. More complex simulations are needed to confirm the calculations in order to optimize the design of the pipe system and the RTG. Furthermore the heat transfer of the top of the RTG was not calculated further since there are too many unknowns in the current state of the mission development. First of all the heating of the penetrator will depend greatly on the thickness of the ice-layer and whether the heat from the RTG's in the relay systems can be used to heat up the top of the penetrator for at least part of the mission. 